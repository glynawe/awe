% algolw.tex -- The 'ALGOL W Language Description', typeset in LaTeX    -*-latex-*- -*-outline-*-
\documentclass[a4paper]{article}


\pdfinfo{
   /Title  (ALGOL W Language Description)
   /Author (Henry Bauer, Sheldon Becker, Susan L. Graham, Edwin Satterthwaite, Richard L. Sites)
}

\usepackage{enumerate}
\usepackage{multind}   
\usepackage{listings}
\usepackage{syntax}
\usepackage{slashbox}
\usepackage{longtable}
\usepackage{threeparttable}
\usepackage[mathscr]{euscript}

% (In Debian Squeeze these packages are available after installing the
% texlive-latex-extra package. You may need to download slashbox.sty
% and multind.sty from CTAN https://www.ctan.org/)

% Note: there are \newpage commands in this document to prevent
% examples and tables from being split up, and hard column widths for
% the syntax diagrams. These will matter if you change the font or
% paper size.

\def\ALGOLW{\textsc{Algol W}}

% - - - - - - - - - - - - - - - - - - - - - - - - - - - - - - - - - - - - - - - - - - - - - - - - - - -
% Table of contents.

\setcounter{secnumdepth}{4}  %Allow numbered paragraphs.
\setcounter{tocdepth}{3}

% A \quietsubsubsection is numbered but does not go into the contents page:
\def\quietsubsubsection#1{\addtocounter{subsubsection}{1}\subsubsection*{\thesubsubsection. #1}}

% These subsubsections are too short and numerous to go into the contents page:
\def\Syntax{\quietsubsubsection{Syntax}}
\def\Semantics{\quietsubsubsection{Semantics}}
\def\Examples{\quietsubsubsection{Examples}}
\def\Example{\quietsubsubsection{Example}}
\def\SyntaxUnnumbered{\subsubsection*{Syntax}}
\def\SemanticsUnnumbered{\subsubsection*{Semantics}}

% - - - - - - - - - - - - - - - - - - - - - - - - - - - - - - - - - - - - - - - - - - - - - - - - - - -
% Index.   (Uses the "multind" package.)

\makeindex{special}
\makeindex{syntax}

% "\IndexAlgolW{classification}{code}" indexes an Algol W identifier or
% item of syntax in the source code listing format. (See below.)
\def\IndexAlgolW#1{\index{special}{#1@\A{#1}}}
\def\IndexReservedWord#1{\IndexAlgolW{#1}}
\def\IndexField#1{\IndexAlgolW{#1}}
\def\IndexRecord#1{\IndexAlgolW{#1}}

% Some Algol W procedure and variable names contain underscores,
% but the "listings" package produces unindexable TeX code for 
% identifiers containing underscores. Hence the use of \texttt{x}
% in place of \A!x! here.
\def\IndexAlgolWID#1{\index{special}{#1@\texttt{#1}}}
\def\IndexProcedure#1{\IndexAlgolWID{#1}}
\def\IndexVariable#1{\IndexAlgolWID{#1}}
\def\IndexType#1{}


% - - - - - - - - - - - - - - - - - - - - - - - - - - - - - - - - - - - - - - - - - - - - - - - - - - -
% BNF syntax sections.  (Uses the "syntax" package.)

\renewcommand{\ulitleft}{\normalfont\rmfamily\bfseries\frenchspacing}
\renewcommand{\ulitright}{}
\renewcommand{\litleft}{`\bgroup\normalfont\ttfamily\frenchspacing}
\renewcommand{\litright}{\egroup'}

\grammarparsep1pt     % put very little extra space between production rules
\def\<#1>{\synt{#1}}  % \<x> = format x as a non-terminal production name

%The T symbols that represent type affixes in production rules:
\def\Tee{$\mathscr{T}$ }
\def\TeeSub#1{$\mathscr{T}_{#1}$}


% - - - - - - - - - - - - - - - - - - - - - - - - - - - - - - - - - - - - - - - - - - - - - - - - - - -
% Formatted Algol W source code.  (Uses the "listings" package.)

% Reserved words are bold lowercase because the "syntax" package will
% not allow underlined uppercase.  Section 2.1. Basic Symbols has been 
% altered (two words) to explain the correct substitution.

\lstdefinelanguage{AlgolW}{%
  keywords={true,false,null,integer,real,complex,logical,bits,string,%
    reference,long,,array,procedure,record,begin,end,if,then,else,case,%
    of,div,rem,shr,shl,is,abs,long,short,and,or,goto,go
    to,for,step,until,%
    do,while,comment,value,result,assert,algol,fortran},%
  sensitive=false,% 
  string=[d]{"},%
  keywordcommentsemicolon={end}{else,end}{comment}, %this doesn't seem  to work!
  literate={~}{$\lnot$}{1}{(LNOT)}{$\lnot$}{1},% the (LNOT) is for places where "~" would confuse LaTeX
}[keywords,comments,keywordcomments]

\lstdefinestyle{ReferenceManual}{%
  basicstyle=\small\ttfamily,%
  keywordstyle=\rmfamily\bfseries,%
  flexiblecolumns=true,%
  xleftmargin=\parindent,%
  escapechar=`,%
  mathescape=true}

% These are the not and not-equals symbols used by ALgol W:
\def\NOT{$\lnot$}
\def\NE{${\lnot}{=}$}

% Inline Algol W source code:
\def\A{\lstinline[language=AlgolW,style=ReferenceManual]}

% Inline Algol W reserved word:
\def\R#1{\lstinline[language=AlgolW,style=ReferenceManual]{#1}}

% Inline Algol W reserved word that should be indexed:
\def\Ri#1{\IndexReservedWord{#1}\R{#1}}

% Inline Algol W type name:
% (The names of Algol types are in italics to match nonterminal production names. 
\def\T#1{\textit{#1}}

% Inline Algol W type name, to be indexed:
\def\Ti#1{\IndexType{#1}\T{#1}}

% "Quantities which transcend the ALGOL W language",
% for use in algorithms demonstrating ALGOL W behaviour:
\def\Transcendent#1{\textrm{\textit{\small #1}}}

% Block of Algol W source code:
\lstnewenvironment{AlgolW}{
\lstset{language=AlgolW,style=ReferenceManual}
}{}

% Block of Algol W source code, example code for the manual:
\lstnewenvironment{AlgolWExample}{
\lstset{language=AlgolW,style=ReferenceManual}
}{}

% Algol W built-in procedure declaration:
\lstnewenvironment{ProcedureDeclaration}[2]{
\IndexProcedure{#2}
\lstset{language=AlgolW,style=ReferenceManual}
}{}

% Algol W built-in variable declaration:
\lstnewenvironment{VariableDeclaration}[2]{
\lstset{language=AlgolW,style=ReferenceManual}
\IndexVariable{#2}
}{}


%------------------------------------------------------------------------------------------------------

\begin{document}

\title{\ALGOLW\\Language Description}
\author{by\\Henry Bauer\\Sheldon Becker\\Susan L. Graham\\Edwin Satterthwaite\\Richard L. Sites}{}
\date{June 1972}

\maketitle

\newpage
\tableofcontents


%------------------------------------------------------------------------------------------------------

\newpage
\section{Terminology, Notation and Basic Definitions}

The Reference Language is a phrase structure language, defined by a
formal metalanguage. This metalanguage makes use of the notation and
definitions explained below. The structure of the language \ALGOLW\ is
determined by:

\begin{enumerate}
\item $\mathscr{V}_N$, the set of basic (or \emph{terminal}) constituents of the language,
\item $\mathscr{V}_T$, the set of syntactic entities (or \emph{nonterminal symbols}), and
\item $\mathscr{P}$, the set of syntactic rules (or productions)
\end{enumerate}

% - - - - - - - - - - - - - - - - - - - - - - - - - - - - - - - - - - - - - - - - - - - - - - - - - - -
\subsection{Notation}

A syntactic entity is denoted by its name (a sequence of letters)
closed in the brackets $\langle$ and $\rangle$. A syntactic rule has
the form
\begin{center}
\syntax{<A> ::= x}
\end{center}
where \<A> is a member of $\mathscr{V}_T$, $x$ is any possible sequence of
basic constituents and syntactic entities, simply to be called a
``sequence''. In \ALGOLW, the set $P$ contains the syntactic rule \index{syntax}{\<bar>|textbf} %
\begin{center}
\syntax{<bar> ::= `|'}
\end{center}
implying that \syntax{`|'} is a basic symbol of the language. Adopting
the convention that all references to this basic symbol in other
syntactic rules shall be replaced by \<bar> permits the unambiguous
use subsequently of the notation %
\begin{center}
  \syntax{<A> ::= x | y | \ldots | z}
\end{center}
as an abbreviation for the set of syntactic rules %
\begin{center}
  \syntax{<A> ::= x\\<A> ::= y\\$\ldots$\\<A> ::= z}
\end{center}

\noindent In the syntactic rule %
\index{syntax}{\<empty>}
\begin{center}
  \syntax{<empty> ::= }
\end{center}
the sequence contains zero symbols, i.e. the empty sequence.

% - - - - - - - - - - - - - - - - - - - - - - - - - - - - - - - - - - - - - - - - - - - - - - - - - - -
\newpage
\subsection{Definitions}

\begin{enumerate}
\item A sequence $x$ is said to {\em directly produce} a sequence $y$
  if and only if there exist (possibly empty) sequences $u$ and $w$,
  so that either (i) for some \<A> in $\mathscr{V}_T$, $x = u$\<A>$w$, $y =
  uvw$, and \<A> $::= v$ is a rule in $\mathscr{P}$; or (ii) $x = uw$, $y =
  uvw$ and $v$ is a ``comment'' (see below).

\item A sequence $x$ is said to \emph{produce} a sequence $y$ if and
  only if there exists an ordered set of sequences $s_0, s_1, ...,
  s_n$, so that $x = s_0$, $s_n = y$, and $s_{i-1}$ directly produces
  $s_i$ for all $i = 1, ... , n$.

\item A sequence $x$ is said to be an \ALGOLW\ program if and only if
  its constituents are members of the set $\mathscr{V}_N$, and $x$ can be
  produced from the syntactic entity \<program>.
\end{enumerate}

The sets $\mathscr{V}_N$ and $\mathscr{V}_T - \{$`\verb!|!'$\}$ are
defined through enumeration of their members given throughout in
Section \ref{Sets of Basic Symbols and Syntactic Entities} and
Appendix \ref{Syntactic Entities} of this Report (cf.~also
\ref{Strings}). To provide explanations for the meaning of
\ALGOLW\ programs, the letter sequences denoting syntactic entities
have been chosen to be English words describing approximately the
nature of that syntactic entity or construct. Where words which have
appeared in this manner are used elsewhere in the text, they refer to
the corresponding syntactic definition. Along with these letter
sequences the symbol \Tee may occur. It is understood that this symbol
must be replaced by any one of a finite set of English words (or word
pairs). Unless otherwise specified in the particular section, all
occurrences of the symbol \Tee within one syntactic rule must be
replaced consistently, and the replacing words are
\begin{quote}
\begin{tabular}{ll}
 \T{integer}       & \T{logical} \\
 \T{real}          & \T{bit} \\
 \T{long real}     & \T{string} \\
 \T{complex}       & \T{reference} \\
 \T{long complex}
\end{tabular}
\end{quote}

\noindent For example, the production
\begin{center}
\syntax{<\Tee term> ::= <\Tee factor>}
\end{center}
corresponds to
\begin{center}
\syntax{<integer term> ::= <integer factor>\\
<real term> ::= <real factor>\\
<long real term> ::= <long real factor>\\
<complex term> ::= <complex factor>\\
<long complex term> ::= <long complex factor>}
\end{center}
The production
\begin{center}
\syntax{<\TeeSub0 primary> ::= "long" <\TeeSub1 primary>}
\end{center}
corresponds to
\begin{center}
\syntax{<long real primary> ::= "long" <real primary>\\
<long integer primary> ::= "long" <integer primary> \\
<long complex primary> ::= "long" <complex primary>}
\end{center}

It is recognized that typographical entities exist of lower order than
basic symbols, called characters. The accepted characters are those of
the IBM System 360 EBCDIC code.

The symbol \Ri{comment} followed by any sequence of characters not
containing semicolons, followed by a semicolon, is called a comment.
A comment has no effect on the meaning of a program, and is ignored
during execution of the program. An identifier (cf. \ref{Identifiers})
immediately following the basic symbol \Ri{end} is also regarded as a
comment.

The execution of a program can be considered as a sequence of units of
action. The sequence of these units of action is defined as the
evaluation of expressions and the execution of statements as denoted
by the program. In the definition of the implemented language the
evaluation or execution of certain constructs is either (1) defined by
System 360 operations, e.g., real arithmetic, or (2) left undefined,
e.g., the order of evaluation of arithmetic primaries in expressions,
or (3) said to be \emph{not valid} or \emph{not defined}.


%------------------------------------------------------------------------------------------------------

\newpage
\section{Sets of Basic Symbols and Syntactic Entities}
\label{Sets of Basic Symbols and Syntactic Entities}

\subsection{Basic Symbols}

\index{syntax}{\<bar>}
\begin{quote}
\begin{synshorts}
`A' | `B' | `C' | `D' | `E' | `F' | `G' | `H' | `I' | `J' | `K' | `L' | `M' |
`N' | `O' | `P' | `Q' | `R' | `S' | `T' | `U' | `V' | `W' | `X' | `Y' | `Z' | `_' | \\
\\
`0' | `1' | `2' | `3' | `4' | `5' | `6' | `7' | `8' | `9' | \\
\\
"true" | "false" | `"' | "null" | `#' | `\'' | \\
"integer" | "real" | "complex" | "logical" | "bits" | "string" |
"reference" | "long real" | "long complex" | "array" |
"procedure" | "record" | \\
`,' | `;' | `:' | `.' | `(' | `)' | "begin" | "end" | "if" | "then" | "else" |
"case" | "of" | `+' | `-' | `*' | `/' | `**' | "div" | "rem" | "shr" | "shl" | "is" |
"abs" | "long" | "short" | "and" | "or" | `\NOT' | <bar> | `<=' | `\NE' | `<' |
`=' | `>' | `>=' | `::' | 
`:=' | "goto" | "go to" | "for" | "step" | "until" | "do" | "while" |
"comment" | "value" | "result" | "assert" | "algol" | "fortran"
\end{synshorts}
\end{quote}

%All underlined words, which we call ``reserved words'', 
All bold lowercase words, which we call ``reserved words'',
are represented by the same words in capital letters in an 
actual program, with no intervening blanks.

Adjacent reserved words, identifiers (cf. \ref{Identifiers}) and
numbers must include no blanks and must be separated by at least one
blank space. Otherwise blanks have no meaning and can be used freely
to improve the readability of the program.

\subsection{Syntactic entities}
\label{Syntactic entities}

(See the \emph{Syntactic Entities} appendix on page \pageref{Syntactic Entities}.)


%------------------------------------------------------------------------------------------------------

\newpage
\section{Identifiers}
\label{Identifiers}

\grammarindent1.8in
\index{syntax}{\<\Tee array identifier>|textbf}
\index{syntax}{\<\Tee field identifier>|textbf}
\index{syntax}{\<\Tee function identifier>|textbf}
\index{syntax}{\<\Tee variable identifier>|textbf}
\index{syntax}{\<control identifier>|textbf}
\index{syntax}{\<identifier>|textbf}
\index{syntax}{\<identifier list>|textbf}
\index{syntax}{\<label identifier>|textbf}
\index{syntax}{\<procedure identifier>|textbf}
\index{syntax}{\<record class identifier>|textbf}
\begin{grammar}
<identifier> ::= <letter>
            \alt <identifier> <letter>
            \alt <identifier> <digit>
            \alt <identifier> `_'

<\Tee variable identifier> ::= <identifier>

<\Tee array identifier> ::= <identifier>

<procedure identifier> ::= <identifier>

<\Tee function identifier> ::= <identifier>

<record class identifier> ::= <identifier>

<\Tee field identifier> ::= <identifier>

<label identifier> ::= <identifier>

<control identifier> ::= <identifier>

<letter> ::= `A' | `B' | `C' | `D' | `E' | `F' | `G'
        \alt `H' | `I' | `J' | `K' | `L' | `M' | `N'
        \alt `O' | `P' | `Q' | `R' | `S' | `T' | `U'
        \alt `V' | `W' | `X' | `Y' | `Z'

<digit> ::= `0' | `1' | `2' | `3' | `4' | `5' | `6' | `7' | `8' | `9'

<identifier list> ::= <identifier>
                 \alt <identifier list> `,' <identifier>
\end{grammar}

\SemanticsUnnumbered

Variables, arrays, procedures, record classes and record fields are
said to be quantities.  Identifiers serve to identify quantities, or
they stand as labels, formal parameters or control identifiers.
Identifiers have no inherent meaning, and can be chosen freely in the
reference language.  In an actual program a reserved word cannot be
used as an identifier.

Every identifier used in a program must be defined. This is achieved
through

\begin{enumerate}[(a)]
\item declaration (cf. Section \ref{Declarations}), if the identifier
  identifies a quantity. It is then said to denote that quantity and
  to be a \Tee variable identifier, \Tee array identifier, \Tee
  procedure identifier, \Tee function identifier, record class
  identifier or \Tee field identifier, where the symbol \Tee
  stands for the appropriate word reflecting the type of the declared
  quantity;

\item a label definition (cf. \ref{Blocks}), if the identifier stands
  as a label. It is then said to be a label identifier;

\item its occurrence in a formal parameter list (cf. \ref{Procedure Declarations}). 
  It is then said to be a formal parameter;

\item its occurrence following the symbol \Ri{for} in a for clause
  (cf. \ref{Iterative Statements}).  It is then said to be a control identifier;

\item its implicit declaration in the language. Standard procedures,
  standard functions, and predefined variables (cf. \ref{Standard Procedures} 
  and \ref{Standard Functions and Predeclared Identifiers}) 
  may be considered to be declared in a block containing the program.
\end{enumerate}

The recognition of the definition of a given identifier is determined
by the following rules:

\begin{enumerate}[{Step} 1.]
\item If the identifier is defined by a declaration of a quantity or
  by its standing as a label within the smallest block (cf.
  \ref{Blocks}) embracing a given occurrence of that identifier, then
  it denotes that quantity or label. A statement following a procedure
  heading (cf. \ref{Procedure Declarations}) or a \R{for} clause (cf.
  \ref{Iterative Statements}) is considered to be a block.

\item Otherwise, if that block is a procedure body and if the given
  identifier is identical with a formal parameter in the associated
  procedure heading, then it stands as that formal parameter.

\item Otherwise, if that block is preceded by a \R{for} clause
  and the identifier is identical to the control identifier of that
  \R{for} clause, then it stands as that control identifier.
\end{enumerate}

Otherwise, these rules are applied considering the smallest block
embracing the block which has previously been considered.

If either step 1 or step 2 could lead to more than one definition,
then the identification is undefined.

The scope of a quantity, a label, a formal parameter, or a control
identifier is the set of statements in which occurrences of an
identifier may refer by the above rules to the definition of that
quantity, label, formal parameter or control identifier.

\Examples

\begin{AlgolWExample}
I
PERSON
ELDERSIBLING
X15, X20, X25
NEW_PAGE
\end{AlgolWExample}


%------------------------------------------------------------------------------------------------------

\newpage
\section{Values and Types}
\label{Values and Types}

Constants and variables (cf.  \ref{Variables}) are said to possess a
value.  The value of a constant is determined by the denotation of the
constant.  In the language, all constants (except references) have a
reference denotation (cf. \ref{Numbers} -- \ref{Strings}).  The value
of a variable is the one most recently assigned to that variable. A
value is (recursively) defined as either a simple value or a
structured value (an ordered set of one or more values). Every value
is said to be of a certain type.

\index{general}{Data types} The following types of simple values are distinguished:
\begin{quote}
\begin{tabular}[t]{lp{3.5in}}
\Ti{integer}      & the value is a 32 bit integer, \\
\Ti{real}         & the value is a 32 bit floating point number, \\
\Ti{long real}    & the value is a 64 bit floating point number, \\
\Ti{complex}      & the value is a complex number composed of two numbers of type \T{real}, \\
\Ti{long complex} & the value is a complex number composed of two \T{long real} numbers, \\
\Ti{logical}      & the value is a logical value, \\
\Ti{bits}         & the value is a linear sequence of 32 bits, \\
\Ti{string}       & the value is a linear sequence of at most 256 characters, \\
\Ti{reference}    & the value is a reference to a record.
\end{tabular}
\end{quote}

The following types of structured values are distinguished:
\begin{quote}
\begin{tabular}[t]{lp{3.5in}}
  \Ti{array}  & the value is an ordered set of values, all of identical simple type, \\
  \Ti{record} & the value is an ordered set of simple values.
\end{tabular}
\end{quote}

A procedure may yield a value, in which case it is said to be a
function procedure, or it may not yield a value, in which case it is
called a proper procedure. The value of a function procedure is
defined as the value which results from the execution of the procedure
body (cf. \ref{Function Designators, Semantics}).

Subsequently, the reference denotation of constants is defined.  The
reference denotation of any constant consists of a sequence of
characters.  This, however, does not imply that the value of the
denoted constant is a sequence of characters, nor that it has the
properties of a sequence of characters, except, of course, in the case
of strings.

% - - - - - - - - - - - - - - - - - - - - - - - - - - - - - - - - - - - - - - - - - - - - - - - - - - -
\newpage
\subsection{Numbers}
\label{Numbers}

\Syntax

\grammarindent1.8in
\index{syntax}{\<\Tee constant>|textbf}
\index{syntax}{\<integer constant>|textbf}
\index{syntax}{\<real constant>|textbf}
\index{syntax}{\<long real constant>|textbf}
\index{syntax}{\<complex constant>|textbf}
\index{syntax}{\<long complex constant>|textbf}
\begin{grammar}
<long complex constant> ::= <complex constant> `L'

<complex constant> ::= <imaginary constant>

<imaginary constant> ::= <real constant> `I'
                  \alt  <integer constant> `I'

<long real constant> ::= <real constant> `L'
                  \alt <integer constant> `L'

<real constant> ::=
             <unscaled real>
        \alt <unscaled real> <scale factor>
        \alt <integer constant> <scale factor>
        \alt <scale factor>

<unscaled real> ::=
             <integer constant> `.' <integer constant>
        \alt `.' <integer constant>
        \alt <integer constant> `.'

<scale factor> ::=
             `'' <integer constant>
        \alt `'' <sign> <integer constant>

<integer constant> ::=
             <digit>
        \alt <integer constant> <digit>

<sign> ::= `+' | `-'
\end{grammar}

(Note: a long complex constant may have the \A!I! and \A!L! in either
order in a program, but they must be in the order \A!IL! on data
cards.)

\Semantics

Numbers are interpreted according to the conventional decimal
notation. A scale factor denotes an integral power of 10 which is
multiplied by the unscaled real or integer number preceding it. Each
number has a uniquely defined type. (Note that all \<\Tee constant>s are
unsigned.)

\Examples

\begin{AlgolWExample}
1                 05                    11
0100              1'3                   0.671
3.1416            6.02486'+23           1IL
2.718281828459045235360287L             2.3'-6
\end{AlgolWExample}


% - - - - - - - - - - - - - - - - - - - - - - - - - - - - - - - - - - - - - - - - - - - - - - - - - - -
\newpage
\subsection{Logical Values}
\label{Logical Values}

\Syntax

\IndexReservedWord{true}
\IndexReservedWord{false}
\IndexType{logical}

\grammarindent0pt
\index{syntax}{\<\Tee constant>|textbf}
\index{syntax}{\<logical constant>|textbf}
\begin{grammar}
<logical constant> ::= "true" | "false"
\end{grammar}

% - - - - - - - - - - - - - - - - - - - - - - - - - - - - - - - - - - - - - - - - - - - - - - - - - - -
\subsection{Bit Sequences}
\label{Bit Sequences}

\Syntax

\grammarindent1.2in
\index{syntax}{\<\Tee constant>|textbf}
\index{syntax}{\<bit constant>|textbf}
\begin{grammar}
<bit constant> ::=
      `#' <hex digit>
 \alt <bit constant> <hex digit>

<hex digit> ::=
      `0' | `1' | `2' | `3' | `4' | `5' | `6' | `7'
 \alt `8' | `9' | `A' | `B' | `C' | `D' | `E' | `F'
\end{grammar}

\noindent Note that \syntax{`A' | \ldots | `F'} corresponds to $10_{10} | ... | 15_{10}$.

\Semantics

The number of bits in a bit sequence is 32, or 8 hex digits. The bit
sequence is always represented by a 32 bit word with the specified bit
sequence right justified in the word and zeros filled in on the left.

\Examples

\begin{AlgolWExample}
    #4F `{\rm= 0000 0000 0000 0000 0000 0000 0100 1111}`
    #9  `{\rm= 0000 0000 0000 0000 0000 0000 0000 1001}`
\end{AlgolWExample}


% - - - - - - - - - - - - - - - - - - - - - - - - - - - - - - - - - - - - - - - - - - - - - - - - - - -
\subsection{Strings}
\label{Strings}

\Syntax

\grammarindent1.4in
\index{syntax}{\<\Tee constant>|textbf}
\index{syntax}{\<string constant>|textbf}
\begin{grammar}
<string constant> ::= <string>

<string> ::= `"' <open string> `"'

<open string> ::= <character> | <open string> <character>
\end{grammar}

\Semantics

Strings consist of any sequence of (at least one and at most 256)
characters accepted by the System 360 enclosed by \lit{"}, the string
quote. If the string quote appears in the sequence of characters it
must be immediately followed by a second string quote which is then
ignored.  The number of characters in a string is said to be the
length of the string. The characters accepted by the IBM system 360
are listed in Appendix \ref{Character Encodings}.

\Example

\begin{AlgolWExample}
   "JOHN"
   """"  `{\rm This is the string of length 1 consisting of the string quote.}`
\end{AlgolWExample}


% - - - - - - - - - - - - - - - - - - - - - - - - - - - - - - - - - - - - - - - - - - - - - - - - - - -
\newpage
\subsection{References}
\label{References}

\Syntax

\IndexReservedWord{null}

\grammarindent0pt
\index{syntax}{\<\Tee constant>|textbf}
\index{syntax}{\<reference constant>|textbf}
\begin{grammar}
<reference constant> ::= "null"
\end{grammar}

\Semantics

The reference value \Ri{null} fails to designate a record; if a
reference expression occurring in a field designator (cf.~\ref{Variables}) 
has this value, then the field designator is undefined.


%------------------------------------------------------------------------------------------------------

\newpage
\section{Declarations}
\label{Declarations}

Declarations serve to associate identifiers with the quantities used
in the program, to attribute certain permanent properties to these
quantities (i.e:. type, structure), and to determine their scope.  The
quantities declared by declarations are simple variables, arrays,
procedures and record classes.

Upon exit from a block, all quantities declared or defined within that
block lose their value and significance (cf. \ref{Blocks, Semantics}
and \ref{Goto Statements, Semantics}).

\SyntaxUnnumbered

\grammarindent1.1in
\index{syntax}{\<declaration>|textbf}
\index{syntax}{\<\Tee array declaration>}
\index{syntax}{\<procedure declaration>}
\index{syntax}{\<record class declaration>}
\index{syntax}{\<simple variable declaration>}
\begin{grammar}
<declaration> ::=  <simple variable declaration>
              \alt <\Tee array declaration>
              \alt <procedure declaration>
              \alt <record class declaration>
\end{grammar}


% - - - - - - - - - - - - - - - - - - - - - - - - - - - - - - - - - - - - - - - - - - - - - - - - - - -
\subsection{Simple Variable Declarations}
\label{Simple Variable Declarations}

\IndexReservedWord{integer}
\IndexReservedWord{real}
\IndexReservedWord{long real}
\IndexReservedWord{complex}
\IndexReservedWord{long complex}
\IndexReservedWord{logical}
\IndexReservedWord{bits}
\IndexReservedWord{string}
\IndexReservedWord{reference}

\Syntax

\grammarindent1.1in
\index{syntax}{\<simple type>|textbf}
\index{syntax}{\<simple variable declaration>|textbf}
\index{syntax}{\<integer constant>}
\index{syntax}{\<identifier list>}
\index{syntax}{\<record class identifier>}
\begin{grammar}
<simple variable declaration> ::=  <simple type> <identifier list>

<simple type> ::=
      "integer" | "real" | "long real" | "complex"
 \alt "long complex" | "logical" |
 \alt "bits" | "bits" `(' `32' `)'
 \alt "string" | "string" `(' <integer constant> `)'
 \alt "reference" `(' <record class identifier list> `)'

<record class identifier list> ::=
      <record class identifier>
 \alt <record class identifier list>`,' <record class identifier>
\end{grammar}

\Semantics

Each identifier of the identifier list is associated with a variable
which is declared to be of the indicated type. A variable is called a
simple variable, if its value is simple (cf. Section
\ref{Declarations}). If a variable is declared to be of a certain
type, then this implies that only values which are assignment
compatible with this type (cf. \ref{Assignment Statements, Semantics})
can be assigned to it. It is understood that the value of a variable
is equal to the value of the expression most recently assigned to it.

A variable of type \Ti{bits} is always of length 32 whether or not
the declaration specification is included.

A variable of type \Ti{string} has a length equal to the unsigned
integer in the declaration specification. If the simple type is given
only as \A!string!, the length of the variable is 16 characters.

 A variable of type
\Ti{reference} may refer only to records of the record classes whose
identifiers appear in the record class identifier list of the
reference declaration specification.

\newpage
\Example

\begin{AlgolWExample}
integer I, J, K, M, N
real X, Y, Z
long complex C
logical L
bits G, H
string (10) S, T
reference (PERSON) JACK, JILL
\end{AlgolWExample}


% - - - - - - - - - - - - - - - - - - - - - - - - - - - - - - - - - - - - - - - - - - - - - - - - - - -
\subsection{Array Declarations}
\label{Array Declarations}

\IndexReservedWord{array}

\Syntax

\grammarindent1.6in
\index{syntax}{\<\Tee array declaration>|textbf}
\index{syntax}{\<\Tee expression>}
\index{syntax}{\<identifier list>}
\index{syntax}{\<simple type>}
\begin{grammar}
<\Tee array declaration> ::= <simple type> "array" <identifier list> \\
                           `(' <bound pair list> `)'

<bound pair list> ::= <bound pair>
                  \alt <bound pair list> `,' <bound pair>

<bound pair> ::= <lower bound> `::' <upper bound>

<lower bound> ::= <integer expression>

<upper bound> ::= <integer expression>
\end{grammar}

\Semantics

Each identifier of the identifier list of an array declaration is
associated with a variable which is declared to be of type \T{array}.
A variable of type \T{array} is an ordered set of variables whose type is
the simple type preceding the symbol array. The dimension of the array
is the number of entries in the bound pair list.

Every element of an array is identified by a list of indices. The
indices are the integers between and including the values of the 
lower bound and the upper bound. Every expression in the bound pair list is
evaluated exactly once upon entry to the block in which the
declaration occurs. The bound pair expressions can depend only on
variables and procedures global to the block in which the declaration
occurs. In order to be valid, for every bound pair, the value of the
upper bound must not be less than the value of the lower bound.

\Example

\begin{AlgolWExample}
integer array H(1::100)
real array A, B(1::M, 1::N)
string (12) array STREET, TOWN, CITY (J::K + 1)
\end{AlgolWExample}


% - - - - - - - - - - - - - - - - - - - - - - - - - - - - - - - - - - - - - - - - - - - - - - - - - - -
\newpage
\subsection{Procedure Declarations}

\label{Procedure Declarations}
\IndexReservedWord{procedure}
\label{Procedure Declarations, Syntax}

\Syntax

\grammarindent2.4in
\index{syntax}{\<\Tee function procedure declaration>|textbf}
\index{syntax}{\<formal parameter list>|textbf}
\index{syntax}{\<procedure heading>|textbf}
\index{syntax}{\<proper procedure declaration>|textbf}
\index{syntax}{\<\Tee expression>}
\index{syntax}{\<block body>}
\index{syntax}{\<external reference>}
\index{syntax}{\<formal array parameter>}
\index{syntax}{\<formal type>}
\index{syntax}{\<identifier>}
\index{syntax}{\<identifier list>}
\index{syntax}{\<simple type>}
\index{syntax}{\<statement>}
\begin{grammar}
<procedure declaration> ::=
      <proper procedure declaration>
 \alt <\Tee function procedure declaration>

<proper procedure declaration> ::=
      "procedure" <procedure heading> `;' \\
      <proper procedure body>

<\Tee function procedure declaration> ::=
      <simple type> \\
      "procedure" <procedure heading> `;' \\
      <\Tee function procedure body>

<proper procedure body> ::=
      <statement>
 \alt <external reference>

<\Tee function procedure body> ::=
      <\Tee expression>
 \alt <block body> <\Tee expression> "end"
 \alt <external reference>

<procedure heading> ::=
      <identifier>
 \alt <identifier> \\
      `(' <formal parameter list> `)'

<formal parameter list> ::=
      <formal parameter segment>
 \alt <formal parameter list> `;' \\
      <formal parameter segment>

<formal parameter segment> ::=
      <formal type> <identifier list>
 \alt <formal array parameter>

\end{grammar}
\IndexReservedWord{value}
\IndexReservedWord{result}
\index{syntax}{\<formal type>|textbf}
\index{syntax}{\<identifier list>}
\index{syntax}{\<simple type>}
\begin{grammar}
<formal type> ::=
      <simple type>
 \alt <simple type> "value"
 \alt <simple type> "result"
 \alt <simple type> "value" "result"
 \alt <simple type> "procedure"
 \alt "procedure"

<formal array parameter> ::=
      <simple type> "array" <identifier list> \\
      `(' <dimension specification> `)'

<dimension specification> ::=
      `*'
 \alt <dimension specification> `,' `*'

\end{grammar}
\IndexReservedWord{fortran}
\IndexReservedWord{algol}
\index{syntax}{\<external reference>|textbf}
\index{syntax}{\<string constant>}
\begin{grammar}
<external reference> ::=
      "fortran" <string constant>
 \alt "algol" <string constant>
\end{grammar}

\Semantics
\setcounter{paragraph}{0}
\label{Procedure Declarations, Semantics}

A procedure declaration associates the procedure body with the
identifier immediately following the symbol \Ri{procedure}. The
principal part of the procedure declaration is the procedure body.
Other parts of the block in whose heading the procedure is declared
can then cause this procedure body to be executed or evaluated. A
proper procedure is activated by a procedure statement (cf.
\ref{Procedure Statements}), a function procedure by a function
designator (cf. \ref{Function Designators}). Associated with the
procedure body is a heading containing the procedure identifier and
possibly a list of formal parameters.

\paragraph{Type specification of formal parameters.}  All formal
parameters of a formal parameter segment are of the same indicated
type.  The type must be such that the replacement of the formal
parameter by the actual parameter of this specified type leads to
correct \ALGOLW\ expressions and statements 
(cf. \ref{Procedure Statements, Semantics}).

\paragraph{} \label{Value/Result Semantics}The effect of the symbols
\R{value} and \R{result} appearing in a formal type is explained by
the following rule, which is applied to the procedure body before the
procedure is invoked:

\begin{enumerate}
\item The procedure body is enclosed by the symbols \R{begin} and
  \R{end};

\item For every formal parameter whose formal type contains the symbol
  \R{value} or \R{result} (or both),

  \begin{enumerate}
  \item a declaration followed by a semicolon is inserted after the
    first \A!begin! of the procedure body, with a simple type as
    indicated in the formal type, and with an identifier different
    from any identifier valid at the place of the declaration.

  \item throughout the procedure body, every occurrence of the formal
    parameter identifier is replaced by the identifier defined in step 2(a);
  \end{enumerate}

\item If the formal type contains the symbol \Ri{value}, an assignment
  statement (cf. \ref{Assignment Statements}) followed by a semicolon
  is inserted after the declarations in the outermost block of the
  procedure body. Its left part contains the identifier defined in
  step 2a, and its expression consists of the formal parameter
  identifier. The symbol \R{value} is then deleted;

\item If the formal type contains the symbol \Ri{result}, an
  assignment statement preceded by a semicolon is inserted before the
  symbol \A!end! which terminates the procedure body. 
  Its left part contains the formal parameter
  identifier, and its expression consists of the identifier defined in
  step 2a. The symbol \R{result} is then deleted.
\end{enumerate}

\paragraph{Specification of array dimensions.}  The number of \A!*!'s
appearing in the formal array specification is the dimension of the
array parameter.

\paragraph{External references.} Use of an external reference as a
procedure body indicates that the actual procedure body is specified
by the environment in which the program is to be executed. The
information in the external reference is used to locate and interpret
that procedure body. The details of such use depend on the specific
environment.

\newpage
\Examples

\begin{AlgolWExample}
procedure INCREMENT; X := X+1
\end{AlgolWExample}

\begin{AlgolWExample}
real procedure MAX (real value X, Y);
    if X < Y then Y else X
\end{AlgolWExample}

\begin{AlgolWExample}
procedure COPY (real array U, V (*,*); integer value A, B);
    for I := 1 until A do
        for J := 1 until B do U(I,J) := V(I,J)
\end{AlgolWExample}

\begin{AlgolWExample}
real procedure HORNER (real array A (*); integer value N; real value X);
    begin real S; S := 0;
        for I := 0 until N do S := S * X + A(I);
        S
    end
\end{AlgolWExample}

\begin{AlgolWExample}
reference (PERSON) procedure YOUNGESTUNCLE (reference (PERSON) R);
    begin reference (PERSON) P, M;
        P := YOUNGESTOFFPRING (FATHER (FATHER (R)));
        while (P ~= null) and (~ MALE (P)) or (P = FATHER(R)) do
            P := ELDERSIBLING (P);
        M := YOUNGESTOFFSPRING (MOTHER (MOTHER (R)));
        while (M ~= null) and (~ MALE (M)) do
            M := ELDERSIBLING (M);
        if P = null then M else
        if M = null then P else
        if AGE(P) < AGE(M) then P else M
    end
\end{AlgolWExample}

\label{PLOTSUBROUTINE}
\begin{AlgolWExample}
procedure PLOTSUBROUTINE (integer value I); fortran "PLOTSUB"
\end{AlgolWExample}


% - - - - - - - - - - - - - - - - - - - - - - - - - - - - - - - - - - - - - - - - - - - - - - - - - - -
\subsection{Record Class Declarations}
\label{Record Class Declarations}

\Syntax

\IndexReservedWord{record}
\grammarindent1.8in
\index{syntax}{\<record class declaration>|textbf}
\index{syntax}{\<identifier>}
\index{syntax}{\<simple variable declaration>}
\begin{grammar}
<record class declaration> ::=
      "record" <identifier> `(' <field list> `)'

<field list> ::=
      <simple variable declaration>
 \alt <field list> `;' <simple variable declaration>
\end{grammar}

\Semantics

A record class declaration serves to define the structural properties
of records belonging to the class. The principal constituent of a
record-class declaration is a sequence of simple variable declarations
which define the fields and their simple types for the records of this
class and associate identifiers with the individual fields.  A record
class identifier can be used in a record designator
(cf. \ref{Arithmetic Expressions}) to construct a new record of the
given class.

\newpage
\Examples

\begin{AlgolWExample}
record NODE (reference (NODE) LEFT, RIGHT)
\end{AlgolWExample}

\begin{AlgolWExample}
record PERSON ( 
    string NAME; 
    integer AGE; 
    logical MALE;
    reference (PERSON) FATHER, MOTHER, YOUNGESTOFFSPRING, ELDERSIBLING 
);
\end{AlgolWExample}


%------------------------------------------------------------------------------------------------------

\newpage
\section{Expressions}
\label{Expressions}

Expressions are rules which specify how new values are computed from
existing ones. These new values are obtained by performing the
operations indicated by the operators on the values of the operands.
The operands are either constants, variables or function designators,
or other expressions, enclosed by parentheses if necessary.  The
evaluation of operands other than constants may involve smaller units
of action such as the evaluation of other expressions or the execution
of statements. The value of an expression between parentheses is
obtained by evaluating that expression. If an operator has two
operands, then these operands may be evaluated in any order with the
exception of the logical operators discussed in \ref{Logical
  Operators}.

\label{Precedence of Operators}

Expressions are distinguished by a type and precedence level, the
former depending on the types of the operands and the latter resulting
from the precedence hierarchy imposed upon operators in the syntactic
rules which follow. The syntactic entities naming different kinds of
expression in these rules display these attributes, the word
``expression'' prefixed by the a type and, usually, postfixed by an
integer indicating the precedence level. (higher precedence is implied
by increasing magnitude of this integer.) The operators and their
precedence levels are:

\begin{center}
\begin{tabular}{ll}
\textbf{Level} & \textbf{Operators}\\
    1       &  \R{or} \\
    2       &  \R{and} \\
    3       &  \A!~! \\
    4       &  \A!<!, \A!<=!, \A!=!, \A!~=!, \A!>=!, \A!>!, \R{is} \\
    5       &  \A!+!, \A!-! \\
    6       &  \A!*!, \A!/!, \R{div}, \R{rem} \\
    7       &  \R{shl}, \R{shr}, \A!**! \\
    8       &  \R{long}, \R{short}, \R{abs} \\
\end{tabular}
\end{center}

When the types allow an operator at level \emph{i} to be applied to
operands, the resulting expression, which which belongs to the
syntactic class \<\Tee expression i> has the intuitive meaning given
in the second column of the following table.

\begin{center}
\begin{tabular}{lll}
\textbf{Syntactic Entity} & \textbf{Meaning} & \textbf{Definitions}\\
\<\Tee expression 1> & disjunction &  \ref{Expressions}, \ref{Logical Expressions}, \ref{Bit Expressions} \\
\<\Tee expression 2> & conjunction &  \ref{Expressions}, \ref{Logical Expressions}, \ref{Bit Expressions} \\
\<\Tee expression 3> & negation &  \ref{Expressions}, \ref{Logical Expressions}, \ref{Bit Expressions} \\
\<\Tee expression 4> & relation &  \ref{Expressions}, \ref{Logical Expressions} \\
\<\Tee expression 5> & sum &  \ref{Expressions}, \ref{Arithmetic Expressions} \\
\<\Tee expression 6> & term &  \ref{Expressions}, \ref{Arithmetic Expressions} \\
\<\Tee expression 7> & factor &  \ref{Expressions}, \ref{Arithmetic Expressions}, \ref{Bit Expressions} \\
\<\Tee expression 8> & primary &  \ref{Expressions}, \ref{Arithmetic Expressions}, \ref{String Expressions} \\
\end{tabular}
\end{center}

The third column of the table indicates sections where definitions of
these syntactic entities occur.

\newpage

Throughout section \ref{Expressions} and its subsections the symbol
\Tee has to be replaced consistently as described in Section 1, and
where the triplets \TeeSub0, \TeeSub1, \TeeSub2 have to be either all
three replaced by the same one of the words
\begin{quote}
    \T{logical}\\
    \T{bit}\\
    \T{string}\\
    \T{reference}
\end{quote}
or (subject to specification to the contrary) in accordance with the
following ``triplet rules''.

\begin{enumerate}
\item Given the qualities (\T{integer}, \T{real} or \T{complex}) of
  \TeeSub1 and \TeeSub2, the corresponding quality of \TeeSub0 is
  given in the table
  \begin{center}
    \begin{it}
      \begin{tabular}{l|lll}
        \backslashbox{\TeeSub1}{\TeeSub2} & integer   & real    & complex \\
        \hline
        integer & integer   & real    & complex \\
        real    & real      & real    & complex \\
        complex & complex   & complex & complex
      \end{tabular}
    \end{it}
  \end{center}
\item \TeeSub0 has the quality \T{long} either if both \TeeSub1 and
  \TeeSub2 have that quality, or if one has the quality \T{long} and
  the other is \T{integer}.
\end{enumerate}

\SyntaxUnnumbered

\grammarindent1.5in
\index{syntax}{\<\Tee block expression>|textbf}
\index{syntax}{\<\Tee expression>|textbf}
\index{syntax}{\<\Tee expression 1>|textbf}
\index{syntax}{\<\Tee expression 2>|textbf}
\index{syntax}{\<\Tee expression 3>|textbf}
\index{syntax}{\<\Tee expression 4>|textbf}
\index{syntax}{\<\Tee expression 5>|textbf}
\index{syntax}{\<\Tee expression 6>|textbf}
\index{syntax}{\<\Tee expression 7>|textbf}
\index{syntax}{\<\Tee expression 8>|textbf}
\index{syntax}{\<\Tee constant>}
\index{syntax}{\<block body>}
\index{syntax}{\<conditional \Tee expression>}
\begin{grammar}
<\Tee expression> ::= <\Tee expression 1> | <conditional \Tee expression>

<\Tee expression 1> ::= <\Tee expression 2>

<\Tee expression 2> ::= <\Tee expression 3>

<\Tee expression 3> ::= <\Tee expression 4>

<\Tee expression 4> ::= <\Tee expression 5>

<\Tee expression 5> ::= <\Tee expression 6>

<\Tee expression 6> ::= <\Tee expression 7>

<\Tee expression 7> ::= <\Tee expression 8>

<\Tee expression 8> ::= <\Tee constant> | `(' <\Tee expression> `)' | <\Tee block expression> 

<\Tee block expression>  ::= <block body> <\Tee expression> "end"
\end{grammar}

\SemanticsUnnumbered

There are 8 levels of precedence; an expression at one level of
precedence is a valid expression at a lower level of precedence.

A block expression introduces a new level of nomenclature and
specifies the execution of statements in the block body as described
for blocks (cf.\ref{Blocks, Semantics}). After execution of of the
block body, the final expression is evaluated and the value of that
expression becomes the value of the entire block expression.


% - - - - - - - - - - - - - - - - - - - - - - - - - - - - - - - - - - - - - - - - - - - - - - - - - - -
\subsection{Variables}
\label{Variables}

\Syntax

\grammarindent1.6in
\index{syntax}{\<\Tee array designator>|textbf}
\index{syntax}{\<\Tee field designator>|textbf}
\index{syntax}{\<\Tee variable>|textbf}
\index{syntax}{\<simple \Tee variable>|textbf}
\index{syntax}{\<subscript>|textbf}
\index{syntax}{\<\Tee array identifier>}
\index{syntax}{\<\Tee expression>}
\index{syntax}{\<\Tee field identifier>}
\index{syntax}{\<\Tee variable identifier>}
\index{syntax}{\<substring designator>}
\begin{grammar}
<simple \Tee variable> ::= <\Tee variable identifier>
                      \alt <\Tee field designator> 
                      \alt <\Tee array designator>

<\Tee variable> ::= <simple \Tee variable>

<string variable> ::= <substring designator>

<\Tee field designator> ::= <\Tee field identifier> `(' <reference expression> `)'

<\Tee array designator> ::= <\Tee array identifier> `(' <subscript list> `)'

<subscript list> ::= <subscript>
                 \alt <subscript list> `,' <subscript>

<subscript> ::= <integer expression>
\end{grammar}

\Semantics

An array designator denotes the variable whose indices are the current
values of the expressions in the subscript list. The value of each
subscript must lie within the declared bounds for that subscript
position.

A field designator designates a field in the record referred to by its
reference expression. The simple type of the field designator is
defined by the declaration of that field identifier in the record
class designated by the reference expression of the field designator
(cf.~\ref{Record Class Declarations}).

\Examples

\begin{AlgolWExample}
X       A(I)     M(I+J, I-J)
FATHER(JACK)     MOTHER(FATHER(JILL))
\end{AlgolWExample}


% - - - - - - - - - - - - - - - - - - - - - - - - - - - - - - - - - - - - - - - - - - - - - - - - - - -
\newpage
\subsection{Function Designators}
\label{Function Designators}

\Syntax

\grammarindent1.7in
\index{syntax}{\<\Tee function designator>|textbf}
\index{syntax}{\<\Tee function identifier>}
\index{syntax}{\<actual parameter list>}
\begin{grammar}
<\Tee function designator> ::=  <\Tee function identifier>
                         \alt <\Tee function identifier> `(' <actual parameter list> `)'
\end{grammar}

\Semantics
\label{Function Designators, Semantics}

A function designator defines a value which can be obtained by a
process performed in the following steps:

\begin{description}
\item[Step 1.]  A copy is made of the body of the function procedure
  whose procedure identifier is given by the function designator and
  of the actual parameters of the latter.

\item[Steps 2, 3, 4.]  As specified in \ref{Procedure Statements, Semantics}.

\item[Step 5.]  The copy of the function procedure body, modified as
  indicated in steps 2--4, is executed. Execution of the expression
  which constitutes or is part of the modified procedure body consists
  of evaluation of that expression, and the resulting value is the
  value of the function designator.  The simple type of the function
  designator is the simple type in the corresponding function
  procedure declaration.
\end{description}

\Examples

\begin{AlgolWExample}
MAX (X ** 2, Y ** 2)
SUM (I, 100, H(1))
SUM (I, M, SUM (J, N, A(I,J)))
YOUNGESTUNCLE (JILL)
SUM (I, 10, X(1) * Y(1))
HORNER (X, 10, 2.7)
\end{AlgolWExample}


% - - - - - - - - - - - - - - - - - - - - - - - - - - - - - - - - - - - - - - - - - - - - - - - - - - -
\subsection{Arithmetic Expressions}
\label{Arithmetic Expressions}

\Syntax
\label{Arithmetic Expressions, Syntax}

\IndexReservedWord{div}
\IndexReservedWord{rem}
\IndexReservedWord{shl}
\IndexReservedWord{shr}
\IndexReservedWord{abs}
\IndexReservedWord{long}
\IndexReservedWord{short}
\grammarindent1.7in
\index{syntax}{\<\Tee expression 5>|textbf}
\index{syntax}{\<\Tee expression 6>|textbf}
\index{syntax}{\<\Tee expression 7>|textbf}
\index{syntax}{\<\Tee expression 8>|textbf}
\index{syntax}{\<control identifier>}
\begin{grammar}
<\TeeSub3 expression 5> ::=  `+' <\TeeSub3 expression 6> 
                        \alt `-' <\TeeSub3 expression 6>

<\TeeSub0 expression 5> ::=  <\TeeSub1 expression 5> `+' <\TeeSub2 expression 6>
                        \alt <\TeeSub1 expression 5> `-' <\TeeSub2 expression 6>

<\TeeSub0 expression 6> ::=  <\TeeSub1 expression 6> `*' <\TeeSub2 expression 7>
                        \alt <\TeeSub1 expression 6> `/' <\TeeSub2 expression 7>

<integer expression 6> ::=  <integer expression 6> "div" <integer expression 7>
                       \alt <integer expression 6> "rem" <integer expression 7>

<\TeeSub4 expression 7> ::=  <\TeeSub5 expression 7> `**' <integer expression 8>

<\TeeSub4 expression 8> ::=  "abs"   <\TeeSub5 expression 8>
                        \alt "long"  <\TeeSub5 expression 8>
                        \alt "short" <\TeeSub5 expression 8>

<integer expression 8> ::=  <control identifier>
\end{grammar}

\Semantics
\label{Arithmetic Expressions, Semantics}

An arithmetic expression is a rule for computing a number.  According
to its simple type it is called an integer expression, real
expression, long real expression, complex expression, or long complex
expression.

\setcounter{paragraph}{0}
\paragraph{}  The operators
\A!+!, \A!-!, \A!*!, and \A!/! have the conventional meanings of
addition, subtraction, multiplication and division. 

For the operator \A!*!, the second ``triplet rule'' is modified so
that \TeeSub0 has the quality \T{long} unless both \TeeSub1 and
\TeeSub2 are \T{integer}.

For the operator \A!/!, the ``triplet rules'' except when both
\TeeSub1 \TeeSub2 are \T{integer}, then \TeeSub0 is \T{long real}.

\paragraph{} The operator \A!-! standing as the first symbol of a
expression at priorty level 5 denotes the monadic operation of sign
inversion. The type of the result is the type of the operand. The
operator \A!+!  standing as the first symbol of a simple expression
denotes the monadic operation of identity.

In the relevant syntactic rules of \ref{Arithmetic Expressions, Syntax}, 
every occurrence of the symbol \TeeSub3 must be systematically
replaced by one of the following words (or word pairs):

\begin{quote}
\Ti{integer} \\
\Ti{real}\\
\Ti{long real}\\
\Ti{complex}\\
\Ti{long complex}
\end{quote}

\IndexReservedWord{div}
\paragraph{} The operator \Ri{div} is mathematically defined (for
${\tt B} \not= 0$) as
\begin{AlgolW}
A div B = SGN(A $\times$ B) $\times$ D(abs A, abs B)
\end{AlgolW}
(cf. \ref{Arithmetic Expressions, abs}) \A!A! and \A!B! both must be
integer expressions.

For the purpose of the definition above, \A!SGN! and \A!D! mean

\begin{AlgolW}
integer procedure SGN (integer value A);
    if A < 0 then -1 else 1;

integer procedure D (integer value A, B);
    if A < B then 0 else D(A - B, B) + 1
\end{AlgolW}

\IndexReservedWord{rem}
\paragraph{}  The operator \Ri{rem} (remainder) is mathematically defined as
\begin{AlgolW}
A rem B = A $-$ (A div B) $\times$ B
\end{AlgolW}

\paragraph{} The operator \A!**! denotes exponentiation of the first
operand to the power of the second operand. In the relevant syntactic
rule of \ref{Arithmetic Expressions, Syntax}. the symbols \TeeSub4,
\TeeSub5 are to be replaced by any of the following combinations of words:
\begin{center}
    \begin{tabular}{l|l}
      \TeeSub4 & \TeeSub5 \\
      \hline
      \T{long real}    & \T{integer} \\
      \T{long real}    & \T{real} \\
      \T{long complex} & \T{complex}
    \end{tabular}
\end{center}

\noindent\TeeSub4 has the quality \T{long} whether or not \TeeSub5 has.

\label{Arithmetic Expressions, abs}
\IndexReservedWord{abs}
\paragraph{} The monadic operator \Ri{abs} yields the absolute value
or modulus of the operand. In the relevant syntactic rule of
\ref{Arithmetic Expressions, Syntax}. the symbols \TeeSub4 and
\TeeSub5 have to be replaced by any of the following combinations of words:
\begin{center}
    \begin{tabular}{l|l}
      \TeeSub4 & \TeeSub5 \\
      \hline
      \T{integer} & \T{integer} \\
      \T{real}    & \T{real} \\
      \T{real}    & \T{complex}
    \end{tabular}
\end{center}

\noindent\ If \TeeSub5 has the quality \T{long} then so does \TeeSub5.

\paragraph{Precision of arithmetic.} If the result of an arithmetic
operation is of simple type \T{real}, \T{complex}, \T{long real}, or
\T{long complex} then it value is defined by System/360 arithmetic and
is the mathematically understood result of the operation performed on
operands which may deviate from actual operands.

In the relevant syntactic rules of \ref{Arithmetic Expressions, Syntax} 
the symbols \TeeSub4, \TeeSub5 must be replaced by the following
combinations of words (or word pairs):

\IndexReservedWord{long}

\noindent\ \\Operator \A!long!
\begin{center}
\begin{tabular}{l|l}
  \TeeSub4     & \TeeSub5 \\
  \hline
  \T{long real}    & \T{real} \\
  \T{long real}    & \T{integer} \\
  \T{long complex} & \T{complex} \\
\end{tabular}
\end{center}

\IndexReservedWord{long}
\noindent\ \\Operator \A!short!
\begin{center}
\begin{tabular}{l|l}
  \TeeSub4     & \TeeSub5 \\
  \hline
  \T{real}    & \T{long real}  \\
  \T{complex} & \T{long complex} \\
\end{tabular}
\end{center}

\Examples

\A!C + A(1) * B(1)! \\
\A!EXP(-X/(2 * SIGMA)) / SQRT (2 * SIGMA)!


% - - - - - - - - - - - - - - - - - - - - - - - - - - - - - - - - - - - - - - - - - - - - - - - - - - -
\subsection{Logical Expressions}
\label{Logical Expressions}

\Syntax
\label{Logical Expressions, Syntax}

In the following rules for \<relation> the symbols \TeeSub6 and
\TeeSub7 must either be identically replaced by any one of the
following words:
\begin{quote}
    \T{bit}\\
    \T{string} \\
    \T{reference}
\end{quote}
or by any of the words from:
\begin{quote}
    \T{complex} \\
    \T{long complex} \\
    \T{real} \\
    \T{long real} \\
    \T{integer}
\end{quote}
and the symbols \TeeSub8 and \TeeSub9 must be identically replaced by
\T{string} or must be replaced by any of \T{real}, \T{long real},
\T{integer}.

\grammarindent1.6in
\index{syntax}{\<\Tee expression 1>|textbf}
\index{syntax}{\<\Tee expression 2>|textbf}
\index{syntax}{\<\Tee expression 3>|textbf}
\index{syntax}{\<equality operator>|textbf}
\index{syntax}{\<inequality operator>|textbf}
\index{syntax}{\<relation>|textbf}
\index{syntax}{\<\Tee expression 4>}
\index{syntax}{\<\Tee expression 5>}
\index{syntax}{\<record class identifier>}
\begin{grammar}
<logical expression 1> ::=
      <logical expression 1>
 \alt <logical expression 1> "or" <logical expression 2>

<logical expression 2> ::=
      <logical expression 2>
 \alt <logical expression 2> "and" <logical expression 3>

<logical expression 3> ::= `\NOT' <logical expression 4>

<relation> ::=
      <\TeeSub6 expression 5> <equality operator>   <\TeeSub7 expression 5>
 \alt <\TeeSub8 expression 5> <inequality operator> <\TeeSub9 expression 5>
 \alt <reference expression 5> "is" <record class identifier>

<equality operator> ::= `=' | `\NE'

<inequality operator> ::= `<' | `<=' | `>=' | `>'

\end{grammar}

\Semantics

A logical expression is a rule for computing a logical value.

\setcounter{paragraph}{0}
\paragraph{} The relational operators represent algebraic ordering for
arithmetic arguments and EBCDIC ordering for string arguments. If two
strings of unequal length are compared, the shorter string is
considered to be extended to the length of the longer (for comparision
only) by appending blanks to the right. The relational operators yield
the logical value \Ri{true} if the relation is satisfied for the
values of the two operands; \Ri{false} otherwise.  Two references are
equal if and only if they are both \Ri{null} or both refer to the same
record. The operator \Ri{is} yields the \Ti{logical} value \R{true} if
the reference expression designates a record of the indicated record
class; \R{false} otherwise.  The reference value \R{null} fails to
designate a record of any record class.

\label{Logical Operators}
\IndexReservedWord{and}
\IndexReservedWord{or}
\paragraph{} The operators \A!~! (not), \Ri{and}, and \Ri{or},
operating on logical values, are defined by the following
equivalences:

\begin{quote}
\begin{tabular}{ll}
 \A!~ X!     & \A!if X then false else true! \\
 \A!X and Y! & \A!if X then Y else false! \\
 \A!X or Y!  & \A!if X then true else Y!
\end{tabular}
\end{quote}

\Examples

\begin{AlgolWExample}
P or Q
(X < Y) and (Y < Z)
YOUNGESTOFFSPRING (JACK) 1 = null
FATHER (JILL) is PERSON
\end{AlgolWExample}



% - - - - - - - - - - - - - - - - - - - - - - - - - - - - - - - - - - - - - - - - - - - - - - - - - - -
\newpage
\subsection{Bit Expressions}
\label{Bit Expressions}

\Syntax

\grammarindent1.4in
\index{syntax}{\<\Tee expression 1>|textbf}
\index{syntax}{\<\Tee expression 2>|textbf}
\index{syntax}{\<\Tee expression 3>|textbf}
\index{syntax}{\<\Tee expression 7>|textbf}
\index{syntax}{\<\Tee expression 4>}
\index{syntax}{\<\Tee expression 8>}
\begin{grammar}
<bit expression 1> ::= <bit expression 1> "or" <bit expression 2>

<bit expression 2> ::= <bit expression 1> "and" <bit expression 3>

<bit expression 3> ::= `\NOT' <bit expression 4>

<bit expression 7> ::= <bit expression 7> "shl" <integer expression 8>
                   \alt <bit expression 7> "shr" <integer expression 8>

\end{grammar}

\Semantics

A bit expression is a rule for computing a bit sequence.

\IndexReservedWord{and}
\IndexReservedWord{or}
The operators \R{and}, \R{or}, and \A!~! produce a result of type
\Ti{bits}, every bit being dependent on the corresponding bit(s) in
the operand(s) as follows:

\begin{center}
\begin{tabular}{cc|ccc}
\A!X! & \A!Y! & \A!~X! & \A!X and Y! & \A!X or Y! \\
\hline
0          & 0          & 1           & 0                & 0 \\
0          & 1          & 1           & 0                & 1 \\
1          & 0          & 0           & 0                & 1 \\
1          & 1          & 0           & 1                & 1
\end{tabular}
\end{center}

\IndexReservedWord{shl}
\IndexReservedWord{shr}
The operators \Ri{shl} and \Ri{shr} denote the shifting operation to
the left and to the right respectively by the number of bit positions
indicated by the absolute value of the integer primary.  Vacated bit
positions to the right or left respectively are assigned the bit value
0.

\Examples

\begin{AlgolWExample}
G and H or #38
G and ~ (H or G) shr 8
\end{AlgolWExample}


% - - - - - - - - - - - - - - - - - - - - - - - - - - - - - - - - - - - - - - - - - - - - - - - - - - -
\subsection{String Expressions}
\label{String Expressions}

\Syntax

\index{syntax}{\<bar>}
\grammarindent1.5in
\index{syntax}{\<substring designator>|textbf}
\index{syntax}{\<integer constant>}
\index{syntax}{\<\Tee expression>}
\index{syntax}{\<\Tee variable>}
\begin{grammar}
<substring designator> ::= <string variable> \\ `(' <integer expression> <bar> <integer constant> `)'
\end{grammar}

\Semantics

A string expression is a rule for computing a string (sequence of characters).

\setcounter{paragraph}{0}
 
\paragraph{} A substring designator denotes a sequence of characters
of the string designated by the string variable. The integer
expression preceding the \<bar> selects the starting character of the
sequence.  The value of the expression indicates the position in the
string variable.  The value must be greater than or equal to 0 and
less than the declared length of the string variable.  The first
character of the string has position 0. The integer number following
the \<bar> indicates the length of the selected sequence and is the
length of the string expression.  The sum of the integer expression
and the integer number must be less than or equal to the declared
length of the string variable.

\Examples

\begin{AlgolWExample}
string (10) S;
S(4|3)
S(I+J|1)
string (10) array T (1::M,2::N);
T(4,6)(3|5)
\end{AlgolWExample}


% - - - - - - - - - - - - - - - - - - - - - - - - - - - - - - - - - - - - - - - - - - - - - - - - - - -
\subsection{Reference Expressions}
\label{Reference Expressions}

\Syntax

\grammarindent1.8in
\index{syntax}{\<\Tee expression 8>|textbf}
\index{syntax}{\<record designator>|textbf}
\index{syntax}{\<\Tee expression>}
\index{syntax}{\<empty>}
\index{syntax}{\<record class identifier>}
\begin{grammar}
<reference expression 8> ::= <record designator>

<record designator> ::=
      <record class identifier>
 \alt <record class identifier> `(' <expression list> `)'

<expression list> ::= <empty> 
                  \alt     <\Tee expression>
                  \alt <expression list> `,' <\Tee expression>
\end{grammar}

\Semantics

A reference expression is a rule for computing a reference to a
record.

The value of a record designator is the reference to a newly created
record belonging to the designated record class.  If the record
designator contains an expression list, then the length of the list
must equal the number of fields specified in the class declaration. 
Values of nonempty expressions in the expression list are
assigned to the corresponding fields of the new record, and the simple
types of the expressions must be assignment compatible with the simple
types of the record fields (cf.\ref{Assignment Statements, Semantics}).

\Example

\begin{AlgolWExample}
PERSON ("CAROL", 0, false, JACK, JILL, null, YOUNGESTOFFSPRING(JACK))
NODE ( , null)
\end{AlgolWExample}


% - - - - - - - - - - - - - - - - - - - - - - - - - - - - - - - - - - - - - - - - - - - - - - - - - - -
\newpage
\subsection{Conditional Expressions}
\label{Conditional Expressions}

\Syntax

\IndexReservedWord{if}
\IndexReservedWord{then}
\IndexReservedWord{else}
\IndexReservedWord{case}
\IndexReservedWord{of}
\grammarindent2.0in
\index{syntax}{\<case clause>|textbf}
\index{syntax}{\<conditional \Tee expression>|textbf}
\index{syntax}{\<if clause>|textbf}
\index{syntax}{\<\Tee expression>}
\begin{grammar}
<conditional \Tee expression> ::= <case clause> `(' <\Tee expression list> `)'

<conditional \TeeSub0 expression> ::= <if clause> <\TeeSub1 expression> "else" <\TeeSub2 expression>

<\Tee expression list> ::= <\Tee expression>

<\TeeSub0 expression list> ::= <\TeeSub1 expression list> `,' <\TeeSub2 expression>

<if clause> ::= "if" <logical expression> "then"

<case clause> ::= "case" <integer expression> "of"
\end{grammar}

\Semantics

The construction
\begin{center}
  \syntax{<if clause> <\TeeSub1 expression> "else" <\TeeSub2 expression>}
\end{center}
causes the selection and evaluation of an expression on the basis of
the current value of the logical expression contained in the \Ri{if}
clause.  If this value is \R{true}, the expression following the
\R{if} clause is selected; if the value is \R{false}, the expression
following \Ri{else} is selected. If \TeeSub1 and \TeeSub2 are type
\Ti{string}, the length of the resulting expression is the maximum of
the lengths of the component expressions; if necessary, blanks are
appended on the right right of the shorter string. The construction
\begin{center}
  \syntax{<case clause> `(' <\Tee expression list> `)'}
\end{center}
causes the selection of the expression whose ordinal number in the
expression list is equal to the current value of the integer
expression contained in the \A!case! clause. In order that the
\A!case!  expression be defined, the current value of
this expression must be the ordinal number of some expression in the
expression list. type If \Tee is type \Ti{string}, the length of the
resulting expression is the maximum of the lengths of the stings in
the expression list. If necessary, the length of any shorter element is
increased by appending blanks on the right.

\Examples

\begin{AlgolWExample}
X    -1    A*B    COLUMN rem 5    (X+Y)*3    long abs BALANCE
if X=3 then Y+37 else Z*2.1
case I of (3.14, 2.78, 448.9)
case DECODE(C) - 128  of ("A", "B", "C", "D", "E", "F")
\end{AlgolWExample}


%------------------------------------------------------------------------------------------------------

\newpage
\section{Statements}
\label{Statements}

A statement denotes a unit of action. By the execution of a statement
is meant the performance of this unit of action, which may consist of
smaller units of action such as the evaluation of expressions or the
execution of other statements.

\SyntaxUnnumbered

\grammarindent1.5in
\index{syntax}{\<program>|textbf}
\index{syntax}{\<simple statement>|textbf}
\index{syntax}{\<statement>|textbf}
\index{syntax}{\<\Tee assignment statement>}
\index{syntax}{\<\Tee function procedure declaration>}
\index{syntax}{\<assert statement>}
\index{syntax}{\<block>}
\index{syntax}{\<case statement>}
\index{syntax}{\<empty>}
\index{syntax}{\<goto statement>}
\index{syntax}{\<if statement>}
\index{syntax}{\<iterative statement>}
\index{syntax}{\<procedure statement>}
\index{syntax}{\<proper procedure declaration>}
\index{syntax}{\<standard procedure statement>}
\begin{grammar}
<program> ::=
      <statement> `.'
 \alt <proper procedure declaration> `.'
 \alt <\Tee function procedure declaration> `.'

<statement> ::=
      <simple statement>
 \alt <iterative statement>
 \alt <if statement>
 \alt <case statement>

<simple statement> ::=
      <block>
 \alt <\Tee assignment statement>
 \alt <empty>
 \alt <procedure statement>
 \alt <goto statement>
 \alt <standard procedure statement>
 \alt <assert statement>
\end{grammar}

Programs which are procedure declarations cannot be executed directly,
but the corrsponding procedure bodies can form part of the environment
in which other \ALGOLW\ programs are executed (cf.\ref{Procedure Declarations}).


% - - - - - - - - - - - - - - - - - - - - - - - - - - - - - - - - - - - - - - - - - - - - - - - - - - -
\subsection{Blocks}
\label{Blocks}

\Syntax

\grammarindent1.4in
\index{syntax}{\<block>|textbf}
\index{syntax}{\<block body>|textbf}
\index{syntax}{\<label definition>|textbf}
\index{syntax}{\<declaration>}
\index{syntax}{\<identifier>}
\index{syntax}{\<statement>}
\begin{grammar}
<block> ::=
      <block body> <statement> "end"

<block body> ::=
      <block head>
 \alt <block body> <statement> `;'
 \alt <block body> <label definition>

<block head> ::=
      "begin" 
 \alt <block head> <declaration> `;'

<label definition> ::=
      <identifier> `:'
\end{grammar}

\Semantics

\label{Blocks, Semantics}
Every block introduces a new level of nomenclature. This is realized
by execution of the block in the following steps:
\begin{description}
\item[Step 1.] If an identifier, say \A!A!, defined in the block head
  or in a  label definition of the block body is
  already defined at the place from which the block is entered, then
  every occurrence of that identifier, A, within the block except for
  occurrence in array bound expressions is systematically replaced by
  another identifier, say $\texttt{A}'$, which is defined neither within
  the block nor at the place from which the block is entered.
\item[Step 2.] If the declarations of the block contain array bound
  expressions, then these expressions are evaluated.
\item[Step 3.] Execution of the statements contained in the block body
  begins with the execution of the first statement following the block
  head.
\end{description}

After execution of the last statement of the block body (unless it is
a goto statement) a block exit occurs, and the statement following the
entire block is executed.

\Example

\begin{AlgolWExample}
begin real U;
      U := X; X := Y; Y := Z; Z := U
end
\end{AlgolWExample}


% - - - - - - - - - - - - - - - - - - - - - - - - - - - - - - - - - - - - - - - - - - - - - - - - - - -
\subsection{Assignment Statements}
\label{Assignment Statements}

\Syntax

In the following rules the symbols \TeeSub0 and \TeeSub1 must be
replaced by words as indicated in Section 1, subject to the
restriction that the type \TeeSub1 is assignment compatible with the
type \TeeSub0 as defined in \ref{Assignment Statements, Semantics}.

\grammarindent2.0in
\index{syntax}{\<\Tee assignment statement>|textbf}
\index{syntax}{\<\Tee expression>}
\index{syntax}{\<\Tee variable>}
\begin{grammar}
<\TeeSub0 assignment statement> ::=
      <\TeeSub0 left part> <\TeeSub1 expression>
 \alt <\TeeSub0 left part> <\TeeSub1 assignment statement>

<\Tee left part> ::= <\Tee variable> `:='
\end{grammar}

\Semantics
\label{Assignment Statements, Semantics}

The execution of a simple assignment statement
\begin{center}
\syntax{<assignment statement> ::= <\TeeSub0 left part> <\TeeSub1 expression>}
\end{center}
causes the assignment of the value of the expression to the variable.
If a shorter string is to be assigned to a longer one, the shorter
string is first extended to the right with blanks until the lengths
are equal. In a multiple assignment statement
\begin{center}
\syntax{<assignment statement> ::= <\TeeSub0 left part> <\TeeSub1 assignment statement>}
\end{center}
the assignments are performed from right to left. For each left part
variable, the simple type of the expression or assignment variable
immediately to the right must be assignment compatible with the simple
type of that variable.

A simple type \TeeSub0 is said to be \emph{assignment compatible} with
a simple type \TeeSub1 if either

\begin{enumerate}[(1)]
\item the two types are identical (except that if \TeeSub0 and
  \TeeSub1 are \T{string}, the length of the \TeeSub0 variable
  must be greater than or equal to the length of the \TeeSub1
  expression or assignment), or

\item \TeeSub0 is \T{real} or \T{long real}, and \TeeSub1 is
  \T{integer}, \T{real} or \T{long real}, or

\item \TeeSub0 is \T{complex} or \T{long complex}, and
  \TeeSub1 is \T{integer}, \T{real}, \T{long real},
  \T{complex} or \T{long complex}.
\end{enumerate}

In the case of a reference, the reference to be assigned must refer to
a record of one of the classes specified by the record class
identifiers associated with the reference variable in its declaration.

\Examples

\begin{AlgolWExample}
z := AGE(JACK) := 28
X := Y + abs Z
C := I + X + C
P := X1 ~= Y
\end{AlgolWExample}


% - - - - - - - - - - - - - - - - - - - - - - - - - - - - - - - - - - - - - - - - - - - - - - - - - - -
\subsection{Procedure Statements}
\label{Procedure Statements}

\Syntax

\grammarindent1.9in
\index{syntax}{\<procedure statement>|textbf}
\index{syntax}{\<actual parameter>|textbf}
\index{syntax}{\<actual parameter list>|textbf}
\index{syntax}{\<\Tee subarray designator>|textbf}
\index{syntax}{\<\Tee array identifier>}
\index{syntax}{\<\Tee expression>}
\index{syntax}{\<\Tee function identifier>}
\index{syntax}{\<procedure identifier>}
\index{syntax}{\<statement>}
\index{syntax}{\<subscript>}
\begin{grammar}
<procedure statement> ::=
      <procedure identifier>
 \alt <procedure identifier> \\ `(' <actual parameter list> `)'

<actual parameter list> ::=
      <actual parameter>
 \alt <actual parameter list> `,' <actual parameter>

<actual parameter> ::=
      <\Tee expression>
 \alt <statement>
 \alt <\Tee subarray designator>
 \alt <procedure identifier>
 \alt <\Tee function identifier>

<\Tee subarray designator> ::=
      <\Tee array identifier>
 \alt <\Tee array identifier> \\ `(' <subarray designator list> `)'

<subarray designator list> ::=
      <subscript>
 \alt `*'
 \alt <subarray designator list> `,' <subscript>
 \alt <subarray designator list> `,' `*'
\end{grammar}

\Semantics
\label{Procedure Statements, Semantics}

The execution of a procedure statement is equivalent to a process
performed in the following steps:

\begin{enumerate}[{Step} 1.]
\item A copy is made of the body of the proper procedure whose
  procedure identifier is given by the procedure statement, and of the
  actual parameters of the latter. The procedure statement is replaced
  by the copy of the procedure body.
\item If the procedure body is a block, then a systematic change of
  identifiers in its copy is performed as specified by step 1 of
  \ref{Blocks, Semantics}.
\item The copies of the actual parameters are treated in an undefined
  order as follows: If the copy is an expression
  different from a variable, then it is enclosed by a pair of
  parentheses, or if it is a statement it is enclosed by the symbols
  \R{begin} and \R{end}.
\item In the copy of the procedure body every occurrence of an
  identifier identifying a formal parameter is replaced by the copy of
  the corresponding actual parameter (cf. \ref{Procedure Statements, Semantics, Actual-formal correspondence}). 
  In order for the
  process to be defined, these replacements must lead to correct \ALGOLW\ 
  expressions and statements.
\item The copy of the procedure body, modified as indicated in steps
  2-4, is executed.
\end{enumerate}

\setcounter{paragraph}{0}

\paragraph{Actual-formal correspondence.}

\label{Procedure Statements, Semantics, Actual-formal correspondence}
The correspondence between the actual parameters and the formal
parameters is established as follows: The actual parameter list of the
procedure statement (or of the function designator) must have the same
number of entries as the formal parameter list of the procedure
declaration heading. The correspondence is obtained by taking the
entries of these two lists in the same order.

\paragraph{} The following table summarises the forms of actual
parameters which may be substituted for each kind of formal
parameter's specification.

\begin{quote}
\index{syntax}{\<\Tee expression>}
\index{syntax}{\<\Tee variable>}
\index{syntax}{\<\Tee function identifier>}
\index{syntax}{\<\Tee subarray designator>}
\index{syntax}{\<procedure identifier>}
\index{syntax}{\<statement>}
\begin{tabular}{l|l}
\textbf{Formal type} & \textbf{Actual parameter} \\
\hline
\syntax{<\Tee type>}              & \<\Tee expression> \\
\syntax{<\TeeSub0 type> "value"}  & \<\TeeSub1 expression> \\
\syntax{<\TeeSub1 type> "result"} & \<\TeeSub0 variable> \\
\syntax{<\TeeSub1 type> "value" "result"} & \<\TeeSub2 variable> \\
\syntax{<\Tee type> "procedure"} & \syntax{<\Tee function identifier> | <\Tee expression>} \\
\syntax{"procedure"}             & \syntax{<procedure identifier> | <statement>} \\
\syntax{<\Tee type> "array"}     & \<\Tee subarray designator> \\
\end{tabular}
\end{quote}

\noindent\ The type \TeeSub1 must be assignment compatible with type
\TeeSub0. \TeeSub1 and \TeeSub2 must be mutually assignment compatible.


\paragraph{Subarray designators.} A complete array may be passed to a
procedure by specifying the name of the array if the number of
subscripts of the actual parameter equals the number of subscripts of
the corresponding formal parameter.  If the actual array parameter has
more subscripts than the corresponding formal parameter, enough
subscripts must be specified by integer expressions so that the number
of \A!*!'s appearing in the subarray designator equals the number of
subscripts of the corresponding formal parameter. The subscript
positions of the formal array designator are matched with the
positions with \A!*!'s in the subarray designator in the order they
appear.

\Examples
\begin{AlgolWExample}
INCREMENT
COPY (A, B, M, N)
INNERPRODUCT (IP, N, A(I,*), B(*,J))
\end{AlgolWExample}


% - - - - - - - - - - - - - - - - - - - - - - - - - - - - - - - - - - - - - - - - - - - - - - - - - - -
\newpage
\subsection{Goto Statements}
\label{Goto Statements}

\grammarindent1.4in
\index{syntax}{\<goto statement>|textbf}
\index{syntax}{\<label identifier>}
\begin{grammar}
<goto statement> ::=
      "goto" <label identifier>
 \alt "go to" <label identifier>
\end{grammar}

\Semantics
\label{Goto Statements, Semantics}

An identifier is called a label identifier if it stands as a label.

A \Ri{goto} statement determines that execution of the text be
continued after the label definition of the label identifier. The
identification of that label definition is accomplished in the
following steps:
\begin{enumerate}[{Step} 1.]
\item If some label definition within the most recently activated but
  not yet terminated block contains the label identifier, then this is
  the designated label definition. Otherwise,
\item The execution of that block is considered as terminated and Step
  1 is taken as specified above.
\end{enumerate}


% - - - - - - - - - - - - - - - - - - - - - - - - - - - - - - - - - - - - - - - - - - - - - - - - - - -
\subsection{If Statements}
\label{If Statements}

\Syntax

\grammarindent1.2in
\index{syntax}{\<if clause>|textbf}
\index{syntax}{\<if statement>|textbf}
\index{syntax}{\<\Tee expression>}
\index{syntax}{\<simple statement>}
\index{syntax}{\<statement>}
\begin{grammar}
<if statement> ::=
      <if clause> <statement>
 \alt <if clause> <simple statement> "else" <statement>

<if clause> ::= "if" <logical expression> "then"
\end{grammar}

\Semantics

The execution of \Ri{if} statements causes certain statements to be
executed or skipped depending on the values of specified logical
expressions. An \R{if} statement of the form
\begin{center}
\syntax{<if clause> <statement>}
\end{center}
is executed in the following steps:
\begin{enumerate}[{Step} 1.]
\item The \Ti{logical} expression in the \R{if} clause is evaluated.
\item If the result of Step 1 is true, then the statement following
  the \R{if} clause is executed. Otherwise Step 2 causes no action to
  be taken at all.
\end{enumerate}

An \R{if} statement of the form
\begin{center}
\syntax{<if clause> <simple statement> "else" <statement>}
\end{center}
is executed in the following steps:
\begin{enumerate}[{Step} 1.]
\item The \Ti{logical} expression in the \R{if} clause is evaluated.
\item If the result of Step 1 is true, then the simple statement
  following the \R{if} clause is executed. Otherwise the statement
  following \Ri{else} is executed.
\end{enumerate}

\Examples
\begin{AlgolWExample}
if X = Y then goto L
if X < Y then U := X else if Y < Z then U := Y else V := Z
\end{AlgolWExample}


% - - - - - - - - - - - - - - - - - - - - - - - - - - - - - - - - - - - - - - - - - - - - - - - - - - -
\newpage
\subsection{Assert Statements}
\label{Assert Statements}
\IndexType{logical}

\Syntax

\grammarindent0pt
\index{syntax}{\<assert statement>|textbf}
\index{syntax}{\<\Tee expression>}
\begin{grammar}
<assert statement> ::= "assert" <logical expression>
\end{grammar}

\Semantics

The \Ri{assert} statement is equivalent to the \R{if} statement:
\begin{quote}
  \A!if (!\<logical expression>\A!) then! \Transcendent{endexecution}
\end{quote}
where \Transcendent{endexecution} signifies a procedure which
terminates the execution of an \ALGOLW\ program. The assert statement
can be used both as a debugging aid (asserting conditions which should
be true, but may not be if a bug exists), and as a program
documentation aid.


% - - - - - - - - - - - - - - - - - - - - - - - - - - - - - - - - - - - - - - - - - - - - - - - - - - -
\subsection{Case Statements}
\label{Case Statements}

\Syntax

\grammarindent1.3in
\index{syntax}{\<case clause>|textbf}
\index{syntax}{\<case statement>|textbf}
\index{syntax}{\<\Tee expression>}
\index{syntax}{\<statement>}
\begin{grammar}
<case statement> ::= <case clause> "begin" <statement list> "end"

<statement list> ::= <statement>
                \alt <statement list> `;' <statement>

<case clause> ::= "case" <integer expression> "of"
\end{grammar}

\Semantics

The execution of a \Ri{case} statement proceeds in the following
steps:
\begin{enumerate}[Step 1.]
\item The expression of the case clause is evaluated.
\item The statement whose ordinal number in the statement list is
  equal to the value obtained in Step 1 is executed. In order that the
  case statement be defined, the current value of the expression in
  the case clause must be the ordinal number of some statement of the
  statement list.
\end{enumerate}

\Examples

\begin{AlgolWExample}
case I of
begin X := X + Y;
      Y := Y + Z;
      Z := Z + X
end
\end{AlgolWExample}
\begin{AlgolWExample}
case J of
begin H(1) := -H(I);
      begin H(I-1) := H(I-1) + H(I); I := I - 1 end;
      begin H(I-1) := H(I-1) * H(I); I := I - 1 end;
      begin H(H(I-1)) := H(I); I := I - 2 end
end
\end{AlgolWExample}


% - - - - - - - - - - - - - - - - - - - - - - - - - - - - - - - - - - - - - - - - - - - - - - - - - - -
\newpage
\subsection{Iterative Statements}
\label{Iterative Statements}

\grammarindent1.6in
\index{syntax}{\<iterative statement>|textbf}
\index{syntax}{\<for clause>|textbf}
\index{syntax}{\<while clause>|textbf}
\index{syntax}{\<\Tee expression>}
\index{syntax}{\<identifier>}
\index{syntax}{\<statement>}
\begin{grammar}
<iterative statement> ::=
      <for clause> <statement>
 \alt <while clause> <statement>

<for clause> ::=
      "for" <identifier> `:=' <initial value> "step" <increment> "until" <limit> "do"
 \alt "for" <identifier> `:=' <initial value> "until" <limit> "do"
 \alt "for" <identifier> `:=' <for list> "do"

<for list> ::= <integer expression>
           \alt <for list> `,' <integer expression>

<initial value> ::= <integer expression>

<increment> ::= <integer expression>

<limit> ::= <integer expression>

<while clause> ::= "while" <logical expression> "do"
\end{grammar}

\Semantics

The iterative statement serves to express that a statement be executed
repeatedly depending on certain conditions specified by a \Ri{for}
clause or a \Ri{while} clause. The statement following the \R{for}
clause or the \R{while} clause always acts as a block, whether it has
the form of a block or not. The value of the control identifier (the
identifier following \Ri{for}) cannot be changed by assignment within
the controlled statement.

\begin{enumerate}[(a)]

\item An iterative statement of the form
\begin{quote}
\IndexReservedWord{step}
\IndexReservedWord{until}
\begin{AlgolW}
for `\<identifier>` := $E_1$ step $E_2$ until $E_3$ do `\<statement>`
\end{AlgolW}
\end{quote}
is exactly equivalent to the block
\begin{quote}
\begin{AlgolW}
begin `\<statement-0>`; `\<statement-1>`; `\ldots`
       `\<statement-I>`; `\ldots`
       `\<statement-N>`
end
\end{AlgolW}
\end{quote}
when in the $I^{th}$ statement every occurrence of the control
identifier is replaced by the value of the expression $(E_1 + I\times\
E_2)$.

The index $N$ of the last statement is determined by $N < (E_3-E_1) /
E2 < N+1$.  If $N < 0$, then it is understood that the sequence is
empty. The expressions $E_1$, $E_2$, and $E_3$ are evaluated exactly
once, namely before execution of \<statement-0>.  Therefore they can
not depend on the control identifier.

\newpage
\item An iterative statement of the form
\begin{quote}
\begin{AlgolW}
for `\<identifier>` := $E_1$ until $E_3$ do `\<statement>`
\end{AlgolW}
\end{quote}
is exactly equivalent to the iterative statement
\begin{quote}
\begin{AlgolW}
for `\<identifier>` := $E_1$ step 1 until $E_3$ do `\<statement>`
\end{AlgolW}
\end{quote}

\item An iterative statement of the form
\begin{quote}
\begin{AlgolW}
for `\<identifier>` := $E_1$, $E_2$, `\ldots`, $E_N$ do `\<statement>`
\end{AlgolW}
\end{quote}
is exactly equivalent to the block
\begin{quote}
\begin{AlgolW}
begin `\<statement-1>` ; `\<statement-2>`; `\ldots`
      `\<statement-I>` ; `\ldots`
      `\<statement-N>`
end
\end{AlgolW}
\end{quote}
when in the $I^{th}$ statement every occurrence of the control
identifier is replaced by the value of the expression $E_I$.

\item
\IndexType{logical}
\IndexReservedWord{while}
An iterative statement of the form
\begin{quote}
\begin{AlgolW}
while $E$ do `\<statement>`
\end{AlgolW}
\end{quote}
is exactly equivalent to the block
\begin{quote}
\begin{AlgolW}
begin
$L$: if $E$ then
      begin `\<statement>` ; goto $L$ end
end
\end{AlgolW}
\end{quote}
where it is understood the $L$ represents an identifier which is not
defined at the place from which the \R{while} statement is entered.

\end{enumerate}


% - - - - - - - - - - - - - - - - - - - - - - - - - - - - - - - - - - - - - - - - - - - - - - - - - - -
\newpage
\subsection{Standard Procedures}
\label{Standard Procedures}

Standard procedures are provided in \ALGOLW\ for the purpose of
communication with the input/output system. A standard procedure
differs from an explicitly declared procedure in that the number and
type of its actual parameters need not be identical in every statement
which invokes the standard procedure identifier
appears.
 
\SyntaxUnnumbered

\grammarindent1.8in
\index{syntax}{\<standard procedure statement>|textbf}
\index{syntax}{\<\Tee expression>}
\index{syntax}{\<\Tee variable>}
\index{syntax}{\<simple statement>}
\begin{grammar}
<standard procedure statement> ::= 
     `READ' `(' <input parameter list> ')'
\alt `READON' `(' <input parameter list> ')'
\alt `READCARD' `(' <input parameter list> ')'
\alt `WRITE' `(' <transput parameter list> ')'
\alt `WRITEON' `(' <transput parameter list> ')'
\alt `IOCONTROL' `(' <transput parameter list> ')'

<input parameter list> ::= 
     <\Tee variable>
\alt <simple statement>
\alt <input parameter list> `,' <\Tee variable>
\alt <input parameter list> `,' <simple statement>

<transput parameter list> ::= 
     <\Tee expression>
\alt <simple statement>
\alt <transput parameter list> `,' <\Tee variable>
\alt <transput parameter list> `,' <simple statement>
\end{grammar}


% - - - - - - - - - - - - - - - - - - - - - - - - - - - - - - - - - - - - - - - - - - - - - - - - - - -
\subsubsection{The Input/Output System}
\label{The Input/Output System}

\ALGOLW\ provides a single legible input stream and a single legible
output stream. These streams are conceived as sequences of records,
each record consisting of a character sequence of fixed length. The
input stream has the logical properties of a sequence of cards in a
card reader; records consist of 80 characters. The output stream has
the logical properties of a sequence of lines on a line printer;
records consist of 132 characters, and the records are grouped into
logical pages.  Each page consists of not less than one nor more than
60 lines.

Input records may be transmitted as strings without analysis.
Alternatively, it is possible to invoke a procedure which will scan
the sequence of records for data items to be interpreted as numbers,
bit sequences, strings, or logical values.  If such analysis is
specified, data items may be reference denotations of the
corresponding constants (cf. Section \ref{Values and Types}).  In
addition, the following forms of arithmetic expressions are acceptable
data items, and the corresponding simple types are those determined by
the rules for expressions (cf. \ref{Arithmetic Expressions}):
\begin{enumerate}[(1)]
\item \syntax{<sign> <\Tee constant>} where \Tee is one of:
  \T{integer}, \T{real}, \T{long real}, \T{complex}, \T{long complex}
\item 
  \syntax{<\TeeSub0 constant> <sign> <\TeeSub1 constant> | <sign> <\TeeSub0 constant> <sign> <\TeeSub1 constant>} 
  where \TeeSub0 is one of: \T{integer}, \T{real}, \T{long real}, \\
  and \TeeSub1 is one of: \T{complex}, \T{long complex}
\end{enumerate}

Data items are separated by one or more blanks. Scanning for data
items initially begins with the first character of the input stream;
after the initial scan, it normally begins with the character
following the one which terminated the most recent previous scan.
Leading blanks are ignored. The scan is terminated by the first blank
following the data item. In the process, new records are fetched as
necessary; character position 80 of one record is considered to be
immediately followed by character position 1 of the next record. There
exist procedures to cause the scanning process to begin with the first
character of a record; if scanning would not otherwise start there, a
new record is fetched.

Output items are assembled into records by an editing procedure.
Items are automatically converted to character sequences and placed in
fields according to the simple type of each item, as described below.
The first field transmitted begins the output stream; thereafter, each
field is normally placed immediately following the most recent
previously transmitted field. If, however, the field corresponding to
an item cannot be placed entirely within a non-empty record, that item
is made the first field of the next record. In addition, there exist
procedures to cause the field corresponding to an item to begin a new
record. Each page group is automatically terminated after 60 records;
procedures are provided for causing earlier termination.


% - - - - - - - - - - - - - - - - - - - - - - - - - - - - - - - - - - - - - - - - - - - - - - - - - - -
\subsubsection{Read Statements}
\label{Read Statements}
\IndexProcedure{READ}
\IndexProcedure{READON}
\IndexProcedure{IOCONTROL}

Both \A!READ! and \A!READON! designate free field input procedures.
Input records are scanned as described in \ref{The Input/Output System}. 
Values on input records are read, matched with the
variables of the actual parameter list in order of appearance, and
assigned to the corresponding variables.  The simple type of each data
item must be assignment compatible with the simple type of the
corresponding variable. For each \A!READ! statement, scanning for the
first data item is caused to begin with the first character of a
record; for a \A!READON! statement, scanning continues from the
previous point of termination as determined by prior use of \A!READ!,
\A!READON!, or \A!IOCONTROL! (cf. \ref{The Input/Output System}).

\A!READCARD! designates a procedure transmitting 80 character input
records without analysis. For each variable of the actual parameter
list, the scanning process is set to begin at the first character of a
record (by fetching a new record if necessary), all 80 characters of
that record are assigned to the corresponding string variable, and
subsequent input scanning is set to begin at the first character of
the next sequential record.


% - - - - - - - - - - - - - - - - - - - - - - - - - - - - - - - - - - - - - - - - - - - - - - - - - - -
\newpage
\subsubsection{Write Statements}
\label{Write Statements}
\IndexProcedure{WRITE}
\IndexProcedure{WRITEON}

\A!WRITE! and \A!WRITEON! designate output procedures with automatic
format conversion.  Values of expressions of the actual parameter list
are converted to character fields which are assembled into output
records in order of appearance (cf. \ref{The Input/Output System}).
For each \A!WRITE! statement, the field corresponding to the first
value is caused to begin an output record; for a \A!WRITEON!
statement, assembly continues from the previous point of termination.

The values of a set of predeclared \emph{editing variables} controls
the field widths and the formats of numerical quantities printed by
the standard \ALGOLW\ output routines. These variables are initialized
to appropriate default settings; their values can be inspected and
modified in the course of the execution of the \ALGOLW\ program. Their
attributes are given in the following table:

\begin{longtable}{llcp{2.3in}}
\textbf{Identifier} & \textbf{Type}    & \textbf{Initial Value} & \textbf{Interpretation} \\
\A!I_W!          & \T{integer}   & \A!14!  & Width of \T{integer} fields. \\
\A!R_FORMAT!     & \T{string(1)} & \A!"F"! & Format of \T{real}, \T{long real}, 
                                             \T{complex} and \T{long complex} fields. \\
\A!R_W!          & \T{integer}   & \A!14!  & Width of real and long real fields; 
                                             width of complex and long complex fields 
                                             $(2 \times {\tt R\_W} + 2)$ \\
\A!R_D!          & \T{integer}   & \A!0!   & Places following the decimal point in 
                                             \T{real}, \T{long real}, \T{complex} and 
                                             \T{long complex} fields. \\
\A!S_W!          & \T{integer}   & \A!2!   & Width of the fields of blanks appended 
                                             to the end of each fields (excluding 
                                             \T{string} fields).
\end{longtable}

\IndexVariable{I\_W}
\IndexVariable{R\_W}
\IndexVariable{R\_D}
\IndexVariable{R\_FORMAT}
\IndexVariable{S\_W}
Values of \A!I_W! and \A!R_W! control the output field widths used for
numerical quantities, in conjunction with the values of of \A!S_W! they
determine the layout of each line of numerical output. Integer
quantities are converted according to a standard format, but three
different formats for the representation of \T{real}, \T{long real},
\T{complex} and \T{long complex} values (strictly, rounded
approximations of these values) are available. For a particular output
value, the actual format is determined by the interrogation of the the
variable \A!R_FORMAT!, which must specify one of the following:

\begin{enumerate}[(1)]
\item \emph{scaled} format (\A!R_FORMAT! = \A!"S"!), in which the
  legible representation takes the form of a normalized mantissa
  followed by an explicit scale factor;
\item \emph{aligned} format (\A!R_FORMAT! = \A!"A"!), in which the
  representation includes an integral part, a fractional part with the
  specified number of digits, but no scale factor;
\item \emph{free-point} format (\A!R_FORMAT! = \A!"F"!), in which the
  representation is chosen to use a specified number of significant
  digits, with the decimal point suitably positioned and with a scale
  factor only if necessary.
\end{enumerate}

Scaled and aligned representations are sometimes said to use
``scientific'' and `` fixed-point'' notation respectively. If scaled
or free-point format is specified, the number of significant digits
printed is given by \A!R_W! - 7. If (but only if) aligned format is
specified, the number of digits following the decimal point is
controlled by the value of \A!R_D!, and the magnitude of the numerical
quantity determines then number of significant digits printed.

The field in which an output item is placed depends on the type of the item, as follows:

\begin{quote}
\begin{tabular}{l|p{3.5in}}
Simple Type           & Field Description \\
\hline
\T{integer}      & Right justified in a field of \A!I_W! characters and followed by \A!S_W! blanks. \\
\T{real}         & Right justified in a field of \A!R_W! characters and followed by \A!S_W! blanks. \\
\T{long real}    & Right justified in a field of \A!R_W! characters and followed by \A!S_W! blanks. \\
\T{complex}      & Right justified in a field of $(2 \times {\tt R\_W} + 2)$ characters and followed by \A!S_W! blanks.\\
\T{long complex} & Right justified in a field of $(2 \times {\tt R\_W} + 2)$ characters and followed by \A!S_W! blanks.\\
\T{logical}      & Right justified in a field of 6 characters followed by  \A!S_W! blanks. \\
\T{string}       & Field length is exactly the length of the string. \\
\T{bits}         & Right justified in a field of 14 characters and followed by \A!S_W! blanks. 
\end{tabular}
\end{quote}

Parameters corresponding to the syntactic class \<simple statement>
are executed as they are encountered in the corresponding output lists;
they cause no values to be transmitted but they can (and normally
should) serve to change the values of the editing variables or the
state of the input/output system. Furthermore, the values 
of the five predeclared editing variables \A!I_W!, \A!R_W!, \A!R_D!,
\A!R_FORMAT! and \A!S_W! are automatically saved at the beginning of
execution of \A!WRITE! or \A!WRITEON! statements and restored at the
end. Thus changes to the values of these variables within a output
statement are localised and can affect only the editing of the
remaining elements of the list, but assignments outside of such a list
can affect all subsequent editing.


% - - - - - - - - - - - - - - - - - - - - - - - - - - - - - - - - - - - - - - - - - - - - - - - - - - -
\subsubsection{Control Statements}
\label{Control Statements}

\IndexProcedure{IOCONTROL}

\A!IOCONTROL! designates a procedure which affects the state of the
input/output system. Argument values with defined effect are listed
below; other values currently have no effect but are explicitly made
available for local use or future expansion.

\begin{longtable}{cp{3.5in}}
\textbf{Value} &  \textbf{Action} (cf. \ref{The Input/Output System}) \endhead
 1 &  Subsequent input scanning begins with the first character of a record. \\
 2 &   Subsequent output assembly begins with 
      the first field of a record. \\
 3 &  Subsequent output assembly begins with 
      the first field of a record which, in turn, begins a new output page. \\
 4 &  Subsequent output assembly has no provision 
      for automatic page skips. \\
 5 &  Subsequent output assembly 
      contains control characters providing automatic page skips. 
      (Initial option.)
\end{longtable}

\Examples

\begin{AlgolWExample}
READCARD(S, LINE(10|180))
WRITE("AVERAGE =", SUM / N)
WRITEON(X(1,J))
IOCONTROL(2)
\end{AlgolWExample}

\IndexVariable{I\_W}
\IndexVariable{R\_W}
\IndexVariable{R\_D}
\IndexVariable{R\_FORMAT}
\IndexVariable{S\_W}
\IndexProcedure{WRITE}
\IndexProcedure{WRITEON}
\IndexProcedure{IOCONTROL}
\noindent Execution of the program,

\begin{AlgolWExample}
begin
  procedure SCALED (integer value N);
    begin R_FORMAT := "S"; R_W := N+7
    end;
  procedure ALIGNED (integer value N,D);
    begin R_FORMAT := "A"; R_W := N+D+1; R_D := D
    end;
  procedure FREE_POINT (integer value N);
    begin R_FORMAT := "F"; R_W := N+7
    end;
  procedure NEW_LINE; IOCONTROL(2);

  FREE_POINT(5); I_W := 2; S_W := 1;

  for I := -1, 0, 32 do
    begin WRITE(S_W := 0, I, ":", NEW_LINE, I/3);
    WRITEON("I  ", ALIGNED(3,2), I/3, "*", SCALED(12), I/3, "*")
    end
end.
\end{AlgolWExample}

\noindent will produce the following lines:

\begin{quote}
\begin{verbatim*}
-1:
    -0.33333I   -0.33 * -3.33333333333'-01 *
 0:
           0I    0.00 *              0     *
32:
      10.667I   10.67 *  1.06666666667'+01 *
\end{verbatim*}
\end{quote}

Note that the setting of \A!S_W! when the corresponding quantity is
transmitted determines the number of trailing blanks; also, edited
valued are always rounded.

\IndexVariable{I\_W}
\IndexVariable{R\_W}
\IndexVariable{R\_D}
\IndexVariable{R\_FORMAT}
\IndexVariable{S\_W}
Any values assigned to \A!I_W!, \A!R_W!, \A!S_W! in excess of 132 are
treated as 132.  In the event that the values of \A!I_W!, \A!R_W!,
\A!R_D!, \A!S_W!, or \A!R_FORMAT!  are erroneous or inconsistent with
the magnitude or precision of the number to be transmitted then
alternative values are used. These values ensure that an approximation
to the number is always transmitted and that not more digits than are
warranted by the precision of then number are transmitted.


%------------------------------------------------------------------------------------------------------

\newpage
\section{Standard Functions and Predeclared Identifiers}
\label{Standard Functions and Predeclared Identifiers}

The \ALGOLW\ environment includes declarations and initialization of
certain procedures and variables which supplement the language
facilities previously described. Such declarations and initialization
are considered to be included in a block which encloses each \ALGOLW\
program (with terminating period eliminated).


% - - - - - - - - - - - - - - - - - - - - - - - - - - - - - - - - - - - - - - - - - - - - - - - - - - -
\subsection{Standard Transfer Functions}
\label{Standard Transfer Functions}

Certain functions for conversion of values from one simple type to
another are provided. These functions are predeclared; the
corresponding implicit declaration headings are listed below:

\begin{ProcedureDeclaration}{Transfer}{TRUNCATE}
integer procedure TRUNCATE (real value X);
comment `\rm the integer $i$ such that $i \leq |{\tt X}| < |i|$ and $i\times {\tt X} \geq 0$ ;
\end{ProcedureDeclaration}

\begin{ProcedureDeclaration}{Transfer}{ENTIER}
integer procedure ENTIER (real value X);
comment `\rm the integer $i$ such that $i \leq {\tt X} < i + 1$ ;
\end{ProcedureDeclaration}

\begin{ProcedureDeclaration}{Transfer}{ROUND}
integer procedure ROUND (real value X);
comment `\rm the value of the expression`;
if X < 0 then TRUNCATE(X-0.5) else TRUNCATE(X+0.5);
\end{ProcedureDeclaration}

\begin{table}
\begin{center}
\begin{tabular}{c|c|c|c}
\A!X!  &  \A!TRUNCATE(X)!  &  \A! ENTIER(X)! &   \A!ROUND(X)! \\
\hline
  2.3  &       2       &      2      &      2 \\
  2.5  &       2       &      2      &      3 \\
  2.7  &       2       &      2      &      3 \\
 -2.3  &      -2       &     -3      &     -2 \\
 -2.5  &      -2       &     -3      &     -3 \\
 -2.7  &      -2       &     -3      &     -3
\end{tabular}
\end{center}
\caption{Values for \A!TRUNCATE!, \A!ENTIER!, and \A!ROUND!}
\end{table}

\begin{ProcedureDeclaration}{Transfer}{EXPONENT}
integer procedure EXPONENT (real value X);
comment `\rm $0$ if ${\tt X} = 0$, otherwise the largest integer $i$ such that $i \le log_{16}(|{\tt X}|) + 1$. This function obtains the exponent used in the S/360 representation of the real number` ;
\end{ProcedureDeclaration}

\begin{ProcedureDeclaration}{Transfer}{ROUNDTOREAL}
real procedure ROUNDTOREAL  (long real value X);
comment `\rm the properly rounded value of {\tt X}` ;
\end{ProcedureDeclaration}

\begin{ProcedureDeclaration}{Transfer}{REALPART}
real procedure REALPART (complex value Z);
comment `\rm the real component of {\tt Z}` ;
\end{ProcedureDeclaration}

\begin{ProcedureDeclaration}{Transfer}{LONGREALPART}
long real procedure LONGREALPART (long complex value Z);
\end{ProcedureDeclaration}

\begin{ProcedureDeclaration}{Transfer}{IMAGPART}
real procedure IMAGPART (complex value Z);
comment `\rm the imaginary component of {\tt Z}` ;
\end{ProcedureDeclaration}

\begin{ProcedureDeclaration}{Transfer}{LONGIMAGPART}
long real procedure LONGIMAGPART (long complex value Z);
\end{ProcedureDeclaration}

\begin{ProcedureDeclaration}{Transfer}{IMAG}
complex procedure IMAG (real value X);
comment `\rm the complex number $0 + {\tt X} i$ ;
\end{ProcedureDeclaration}

\begin{ProcedureDeclaration}{Transfer}{LONGIMAG}
long complex procedure LONGIMAG (long real value X);
\end{ProcedureDeclaration}

\begin{ProcedureDeclaration}{Transfer}{ODD}
logical procedure ODD (integer value N);
comment `\rm the logical value`; N rem 2 = 1;
\end{ProcedureDeclaration}

\begin{ProcedureDeclaration}{Transfer}{BITSTRING}
bits procedure BITSTRING  (integer value N);
comment `\rm two's complement representation of {\tt N}` ;
\end{ProcedureDeclaration}

\begin{ProcedureDeclaration}{Transfer}{NUMBER}
integer procedure NUMBER (bits value X);
comment `\rm integer with two's complement representation {\tt X}` ;
\end{ProcedureDeclaration}

\label{DECODE}
\begin{ProcedureDeclaration}{Transfer}{DECODE}
integer procedure DECODE (string(1) value S);
comment `\rm numeric code for the character {\tt S} (cf. Appendix \ref{EBCDIC})` ;
\end{ProcedureDeclaration}

\begin{ProcedureDeclaration}{Transfer}{CODE}
string(1) procedure  CODE (integer value N);
comment `\rm character with numeric code given by {\bf abs \tt (N \bf rem \tt 256)} (cf. Appendix \ref{EBCDIC}) ;
\end{ProcedureDeclaration}

In the following comments, the significance of characters in the prototype
formats is as follows:
\begin{quote}
\begin{tabular}[t]{lp{\linewidth}}
{\tt D} & decimal digit in a mantissa or integer \\
{\tt E} & decimal digit in an exponent \\
{\tt A} & hexadecimal digit in a mantissa or integer \\
{\tt B} & hexadecimal digit in an exponent \\
${\tt{\pm}}$ & sign (blank for positive mantissa or integer) \\
${\tt{_\sqcup}}$  & blank
\end{tabular}
\end{quote}
Each exponent is unbiased. Decimal exponents represent powers of 10;
hexadecimal exponents represent powers of 16. Each mantissa (except 0)
represents a normalized fraction less than one. Leading zeroes are not
suppressed.

\begin{ProcedureDeclaration}{Transfer}{BASE10}
string(12) procedure BASE10 (real value X);
comment `\rm string encoding of {\tt X} with format ${\tt{_\sqcup}{\pm}EE{\pm}DDDDDDD}$ ;`
\end{ProcedureDeclaration}

\begin{ProcedureDeclaration}{Transfer}{BASE16}
string(12) procedure BASE16 (real value X);
comment `\rm string encoding of {\tt X} with format ${\tt{_\sqcup}{_\sqcup}{\pm}BB{\pm}AAAAAA}$ ;
\end{ProcedureDeclaration}

\begin{ProcedureDeclaration}{Transfer}{LONGBASE10}
string(20) procedure LONGBASE10 (long real value X);
comment `\rm string encoding of {\tt X} with format ${\tt{_\sqcup}{\pm}EE{\pm}DDDDDDDDDDDDDDD}$ ;
\end{ProcedureDeclaration}

\begin{ProcedureDeclaration}{Transfer}{LONGBASE16}
string(20) procedure LONGBASE16 (long real value X);
comment `\rm string encoding of {\tt X} with format ${\tt{_\sqcup}{_\sqcup}{\pm}BB{\pm}AAAAAAAAAAAAAA}$ ;
\end{ProcedureDeclaration}

\begin{ProcedureDeclaration}{Transfer}{INTBASE10}
string(12) procedure INTBASE10 (integer value N);
comment `\rm string encoding of {\tt N} with format ${\tt{_\sqcup}{\pm}DDDDDDDDDD}$ ;`
\end{ProcedureDeclaration}

\begin{ProcedureDeclaration}{Transfer}{INTBASE16}
string(12) procedure INTBASE16 (integer value N);
comment `\rm  unsigned, two's complement string encoding of {\tt N} with format \\ ${\tt{_\sqcup}{_\sqcup}{_\sqcup}{_\sqcup}AAAAAAAA}$ ;`
\end{ProcedureDeclaration}


% - - - - - - - - - - - - - - - - - - - - - - - - - - - - - - - - - - - - - - - - - - - - - - - - - - -
\subsection{Standard Functions of Analysis}
\label{Standard Functions of Analysis}

The following functions of analysis are provided in the system
environment. In some cases, they are partial functions; action for
arguments outside of the allowed domain is described in 8.5. These
functions are predeclared; the corresponding implicit declaration
headings are listed below:

\begin{ProcedureDeclaration}{Analysis}{SQRT}
real procedure SQRT (real value X);
comment `\rm the positive square root of {\tt X}, domain : ${\tt X} \ge 0$ ;
\end{ProcedureDeclaration}

\begin{ProcedureDeclaration}{Analysis}{LONGSQRT}
long real procedure LONGSQRT (long real value X);
comment `\rm the positive square root of {\tt X}, domain : ${\tt X} \ge 0$ ;
\end{ProcedureDeclaration}

\begin{ProcedureDeclaration}{Analysis}{EXP}
real procedure EXP (real value X);
comment `\rm $e^{\tt X}$, domain : ${\tt X} < 174.67$ ;
\end{ProcedureDeclaration}

\begin{ProcedureDeclaration}{Analysis}{LONGEXP}
long real procedure LONGEXP (long real value X);
comment `\rm $e^{\tt X}$, domain : ${\tt X} < 174.67$ ;
\end{ProcedureDeclaration}

\begin{ProcedureDeclaration}{Analysis}{LN}
real procedure LN (real value X);
comment `\rm logarithm of {\tt X} to the base $e$, domain : ${\tt X} \ge 0$ ;
\end{ProcedureDeclaration}

\begin{ProcedureDeclaration}{Analysis}{LONGLN}
long real procedure LONGLN (long real value X);
comment `\rm logarithm of {\tt X} to the base $e$, domain : ${\tt X} \ge 0$ ;
\end{ProcedureDeclaration}

\begin{ProcedureDeclaration}{Analysis}{LOG}
real procedure LOG (real value X);
comment `\rm logarithm of {\tt X} to the base 10, domain : ${\tt X} \ge 0$ ;
\end{ProcedureDeclaration}

\begin{ProcedureDeclaration}{Analysis}{LONGLOG}
long real procedure LONGLOG (long real value X);
comment `\rm logarithm of {\tt X} to the base 10, domain : ${\tt X} \ge 0$ ;
\end{ProcedureDeclaration}

\begin{ProcedureDeclaration}{Analysis}{SIN}
real procedure SIN (real value X);
comment `\rm sine of {\tt X} (radians), domain : $-823550 < {\tt X} < 823550$ ;
\end{ProcedureDeclaration}

\begin{ProcedureDeclaration}{Analysis}{LONGSIN}
long real procedure LONGSIN (long real value X);
comment `\rm sine of {\tt X} (radians), domain : $-3.537{\times}10^{15} < {\tt X} < 3.537{\times}10^{15}$ ;
\end{ProcedureDeclaration}

\begin{ProcedureDeclaration}{Analysis}{COS}
real procedure COS (real value X);
comment `\rm cosine of {\tt X} (radians), domain : $-823550 < {\tt X} < 823550$ ;
\end{ProcedureDeclaration}

\begin{ProcedureDeclaration}{Analysis}{LONGCOS}
long real procedure LONGCOS (long real value X);
comment `\rm cosine of {\tt X} (radians), domain : $-3.537{\times}10^{15} < {\tt X} < 3.537{\times}10^{15}$ ;
\end{ProcedureDeclaration}

\begin{ProcedureDeclaration}{Analysis}{ARCTAN}
real procedure ARCTAN (real value X);
comment `\rm arctangent (radians) of {\tt X}, range : $-\pi/2 < {\tt ARCTAN(X)} < \pi/2$ ;
\end{ProcedureDeclaration}

\begin{ProcedureDeclaration}{Analysis}{LONGARCTAN}
long real procedure LONGARCTAN (long real value X);
comment `\rm arctangent (radians) of {\tt X}, range : $-\pi/2 < {\tt LONGARCTAN(X)} < \pi/2$ ;
\end{ProcedureDeclaration}


% - - - - - - - - - - - - - - - - - - - - - - - - - - - - - - - - - - - - - - - - - - - - - - - - - - -
\subsection{Time Function}
\label{Time Function}

The \ALGOLW\ environment includes a clock which measures elapsed time
since the beginning of program execution. The resolution of that clock
is $1/60$ second. A predeclared function is provided for reading the
clock.

\begin{ProcedureDeclaration}{Miscellaneous}{TIME}
integer procedure TIME (integer value N);
\end{ProcedureDeclaration}

\begin{tabular}{c|ll}
Argument \A!N! & Result & Units \\
\hline
 -1 & time of day & seconds/60 \\
 0 & elapsed execution time & minutes/100 \\
 1 & elapsed execution time & seconds/60 \\
 2 & elapsed execution time & seconds/38400
\end{tabular}
\\

\noindent The result for any other argument is not defined.


% - - - - - - - - - - - - - - - - - - - - - - - - - - - - - - - - - - - - - - - - - - - - - - - - - - -
\newpage
\subsection{Predeclared Variables}
\label{Predeclared Variables}

The following variables are to be considered declared and initialized
by assignment in the conceptual block enclosing the entire \ALGOLW\
program.

The values indicated for \T{real} and \T{long real} quantities are to
be understood as decimal approximations to the actual machine-format
values provided.

\begin{VariableDeclaration}{Input/Output System}{I\_W}
integer I_W;
comment `\rm initialized to 14, controls output field size for integers (cf. \ref{The Input/Output System})` ;
\end{VariableDeclaration}

\begin{VariableDeclaration}{Input/Output System}{R\_W}
integer R_W;
comment `\rm initialized to 14, controls output field size for \T{real}, \T{long real}, \T{complex} and \T{long complex} quantities (cf. \ref{The Input/Output System})` ;
\end{VariableDeclaration}

\begin{VariableDeclaration}{Input/Output System}{R\_D}
integer R_D;
comment `\rm initialized to 0, specifies the number of fraction digits in aligned formats (cf. \ref{The Input/Output System})` ;
\end{VariableDeclaration}

\begin{VariableDeclaration}{Input/Output System}{R\_FORMAT}
integer R_FORMAT;
comment `\rm initialized to \tt "F"\rm, controls output format for \T{real}, \T{long real}, \T{complex} and \T{long complex} quantities (cf. \ref{The Input/Output System})` ;
\end{VariableDeclaration}

\begin{VariableDeclaration}{Input/Output System}{S\_W}
integer S_W; 
comment `\rm initialized to 2, specifies the number of blanks append to to the end of an ouput numeric field (cf. \ref{The Input/Output System})` ;
\end{VariableDeclaration}

\begin{VariableDeclaration}{Implementation limits}{MAXINTEGER}
integer MAXINTEGER;
comment `\rm initialized to 2147483647, the maximum positive \T{integer} allowed by the implementation` ;
\end{VariableDeclaration}

\begin{VariableDeclaration}{Implementation limits}{EPSILON}
real EPSILON;
comment `\rm initialized to $9.536743{\times}10^{-7}$, the largest positive \T{real} number $\epsilon$ provided by the implementation such that $1 + \epsilon = 1$ ;
\end{VariableDeclaration}

\begin{VariableDeclaration}{Implementation limits}{LONGEPSILON}
long real LONGEPSILON;
comment `\rm initialized to $2.22044604925031{\times}10^{-16}$, the largest positive \T{long real} number $\epsilon$ provided by the implementation such that $1 + \epsilon = 1$ ;
\end{VariableDeclaration}

\begin{VariableDeclaration}{Implementation limits}{MAXREAL}
long real MAXREAL;
comment `\rm initialized to $7.23700557733226{\times}10^{75}$, the largest positive \T{long real} number provided by the implementation` ;
\end{VariableDeclaration}

\begin{VariableDeclaration}{Miscellaneous}{PI}
low real PI;
comment `\rm initialized to 3.14159265358979` ;
\end{VariableDeclaration}


% - - - - - - - - - - - - - - - - - - - - - - - - - - - - - - - - - - - - - - - - - - - - - - - - - - -
\newpage
\subsection{Exceptional Conditions}
\label{Exceptional Conditions}

The facilities described below are provided in \ALGOLW\ to allow
detection and control of certain exceptional conditions arising in the
evaluation of arithmetic expressions and standard functions.

\IndexRecord{EXCEPTION} \IndexField{XCPNOTED} \IndexField{XCPLIMIT}
\IndexField{XCPACTION} \IndexField{XCPMARK} \IndexField{XCPMSG}
Implicit declarations:
\begin{AlgolW}
record EXCEPTION (
    logical XCPNOTED;
    integer XCPLIMIT, XCPACTION;
    logical XCPMARK;
    string(64) XCPMSG );
\end{AlgolW}
\IndexVariable{ENDFILE}
\IndexVariable{OVFL}
\IndexVariable{UNFL}
\IndexVariable{DIVZERO}
\IndexVariable{INTOVFL}
\IndexVariable{INTDIVZERO}
\IndexVariable{SQRTERR}
\IndexVariable{EXPERR}
\IndexVariable{LNLOGERR}
\IndexVariable{SINCOSERR}
\begin{AlgolW}
reference(EXCEPTION)
    OVFL, UNFL, DIVZERO,
    INTOVFL, INTDIVZERO,
    SQRTERR, EXPERR, LNLOGERR, SINCOSERR;
\end{AlgolW}

Associated with each exceptional condition which can be processed is a
predeclared reference variable to which references to records of the
class \A!EXCEPTION! can be assigned. Fields of such records control
the processing of exceptions. The association between conditions and
reference variables is as follows:

\begin{longtable}{lp{3.5in}}
\textbf{Reference Variable} &  \textbf{Conditions} \endhead
\A!ENDFILE!     &  end of file detected in input \\
\A!OVFL!        &  \T{real}, \T{long real}, \T{complex}, \T{long complex} (exponent) overflow \\
\A!UNFL!        &  \T{real}, \T{long real}, \T{complex}, \T{long complex} (exponent) underflow \\
\A!DIVZERO!     &  \T{real}, \T{long rea}l, \T{complex}, \T{long complex} division by zero \\
\A!INTOVFL!     &  \T{integer} overflow \\
\A!INTDIVZERO!  &  \T{integer} division by zero \\
\A!SQRTERR!     &  negative argument for \A!SQRT!, \A!LONGSQRT! \\
\A!EXPERR!      &  argument of \A!EXP!, \A!LONGEXP! out of domain (cf. \ref{Standard Functions of Analysis}) \\
\A!LNLOGERR!    &  argument of \A!LN!, \A!LOG!, \A!LONGLN!, \A!LONGLOG! out of domain (cf. \ref{Standard Functions of Analysis}) \\
\A!SINCOSERR!   &  argument of \A!SIN!, \A!COS!, \A!LONGSIN!, \A!LONGCOS! out of domain (cf. \ref{Standard Functions of Analysis})
\end{longtable}

When one of the conditions listed above is detected, the corresponding
reference variable is interrogated, and one of the alternatives
described below is chosen.

\begin{table}
\begin{threeparttable}[b]
\caption{Results for Exceptional Conditions}
\begin{tabular}{l|llll}
condition      & \A!XCPACTION!  & \A!XCPACTION!    & \A!XCPACTION! &  Reference \\
               & $\not=$ 1 or 2 & = 1              & = 2           &  = \R{null} \\
\hline
\A!ENDFILE!\tnote{1}   & 0              & 0                & 0             & 0 \\
\A!OVFL!       & exponent 128   & $\pm$\A!MAXREAL! & 0             & exponent 128\\
               & too small      &                  &               & too small \\
\A!UNFL!       & exponent 128   & 0                & 0             & 0 \\
               & too large      &                  &               &   \\
\A!DIVZERO!    & dividend       & $\pm$\A!MAXREAL! & 0             & dividend \\
\A!INTOVFL!    & true result    & true result      & true result   & true result \\
               & $\pm 2^{32}$   &  $\pm 2^{32}$    &  $\pm 2^{32}$ &  $\pm 2^{32}$ \\
\A!INTDIVZERO! & dividend       & dividend         & dividend      & dividend \\
\A!SQRTERR!    & 0              & \A!SQRT(abs X)!  & 0             & 0 \\
\A!EXPERR!     & 0              & \A!MAXREAL!      & 0             & 0 \\
\A!LNLOGERR!   & 0              & -\A!MAXREAL!     & 0             & 0 \\
\A!SINCOSERR!  & 0              & 0                & 0             & 0
\end{tabular}
\begin{tablenotes}
\item[1] {\small When an \A!ENDFILE! condition occurs on attempting to read a \T{string}, a string of blanks is supplied; for a logical value \A!false! is supplied.}
\end{tablenotes}
\end{threeparttable}
\end{table}

If the value of the reference variable interrogated is \R{null}, the
condition is ignored and execution of the \ALGOLW\ program continues.
In such situations, a value of 0 is returned as the value of a
standard function.  For other conditions the result is that provided
by the underlying IBM System/360 hardware. In determining such a
result, it is to be noted that in those cases in which the detection
of exceptional conditions can be inhibited at the hardware level,
namely integer overflow and exponent underflow, detection is so
inhibited when the corresponding reference is \R{null}.

\newpage
If the value of the reference variable interrogated is not \R{null},
the fields of the record designated by that reference are
interrogated, and processing action is that described by the algorithm
given below in the form of an extended \ALGOLW\ procedure.  Identifiers
in \Transcendent{lower case} represent quantities which transcend the
\ALGOLW\ language; they are explained subsequently.

\begin{AlgolW}
procedure PROCESSEXCEPTION (reference(EXCEPTION) value CONDITION);
   begin
      XCPNOTED(CONDITION) := true;
      XCPLIMIT(CONDITION) := XCPLIMIT(CONDITION) - 1;
      if (XCPLIMIT(CONDITION) < 0) or XCPMARK(CONDITION) then
         WRITE( "***** ERROR NEAR COORDINATE `\Transcendent{nnnn}` - ", 
                 XCPMSG(CONDITION) );
      if XCPLIMIT(CONDITION) < 0 then `\Transcendent{endexecution}` else
      if `\Transcendent{specialcondition}` then
      `\Transcendent{resultant}` := `\Transcendent{default}` else
      `\Transcendent{resultant}` := if XCPACTION(CONDITION) = 1 then `\Transcendent{adjustment}`
                   else if XCPACTION(CONDITION) = 2 then 0L
                   else `\Transcendent{default}`
   end PROCESSEXCEPTION
\end{AlgolW}

This procedure is invoked with the value of the reference variable
appropriate to the condition as actual parameter. The significance of
the special identifiers used is as follows:

\begin{description}
\item[\Transcendent{nnnn}] approximate coordinate of the source code
  which was being executed when the exceptional condition was detected
\item[\Transcendent{endexecution}] procedure to terminate execution of
  the \ALGOLW\ program
\item[\Transcendent{specialcondition}] logical value which is true if,
  and only if, the condition being processed is one of those listed below
\item[\Transcendent{default}] result of the operation or function
  provided by the \ALGOLW\ system prior to invocation of the exception
  processing procedure; this is defined by the hardware for arithmetic
  operations and is the value 0 for standard functions
\item[\Transcendent{resultant}] value to be returned as the result of
  the arithmetic evaluation or standard function invocation
\item[\Transcendent{adjustment}] adjusted result of the operation
  according to the following table:

  \begin{tabular}{l|l}
    \textbf{Special Condition}     &  \textbf{Adjustment} \\
    \hline
    &\\
    Exponent overflow,     &  \\
    division by zero       &  \A!if! \Transcendent{default} \A!< 0 then -MAXREAL else MAXREAL! \\
    &\\
    Exponent underflow     &  \A!0L! \\
    &\\
    \multicolumn{2}{l}{Argument \A!X! out of domain for:} \\
    \A!SQRT!, \A!LONGSQRT! &  \A!SQRT(abs X), LONGSQRT(abs X)! \\
    \A!EXP!,  \A!LONGEXP!  &  \A!MAXREAL! \\
    \A!LN!,  \A!LONGLN!    &  \A!-MAXREAL! \\
    \A!LOG!, \A!LONGLOG!   &  \A!-MAXREAL! \\
    \A!SIN!, \A!LONGSIN!   &  \A!0L! \\
    \A!COS!, \A!LONGCOS!   &  \A!0L! \\
    &\\
    \multicolumn{2}{l}{\A!ENDFILE! on input; according to type:} \\
    numerical              &  \A!0! \\
    \T{logical}            &  \A!false! \\
    \T{string}             &  \A!" "! \\
    \T{bits}               &  \A!#0!
  \end{tabular}
\end{description}

The reference variable \A!UNFL! is initialized by the system to
\R{null}.  All other reference variables listed above are initialized
to references to a special record which is accessible only by the
system. Interrogation of this record by the procedure described above
has the effect of causing the \ALGOLW\ program to be terminated with a
message indicating the type of exception. Any other attempt to access
any field of this record will result in a reference error.

\Example

It is desired to allow up to ten overflows, but to each time replace
the result with \A!MAXREAL! and to print a warning message. \\

\noindent The values needed for this are:

\begin{quote}
\begin{tabular}{lll}
\A!XCPNOTED!  & \R{false} & this will be changed to \R{true} if an overflow occurs. \\
\A!XCPLIMIT!  & \A!10!    & allow up to ten overflows before being cut off. \\
\A!XCPACTION! & \A!1!     & replace the result with \A!+MAXREAL!. \\
\A!XCPMARK!   & \R{true}  & print a message each time an overflow occurs. \\
\A!XCPMSG!    & \A!"..."! & message to be printed.
\end{tabular}
\end{quote}

\noindent The following assignment statement will establish the proper
environment:

\begin{AlgolWExample}
OVFL := EXCEPTION(false, 10, 1, true, "OVERFLOW FIXED UP");
\end{AlgolWExample}


%------------------------------------------------------------------------------------------------------

\newpage
\appendix
\section{Character Encodings}
\label{Character Encodings}
\label{EBCDIC}

The following table presents the correspondence between printable
string characters and their (EBCDIC) integer encodings. This encoding
establishes the ordering relation on characters and thus on strings.
Those characters in parentheses are not available on the line printer.
Integer codes not listed below do not correspond to any established
character. (Also see \A!CODE!, \A!DECODE! on page \pageref{DECODE}.)

\def\cents{\hbox{\rm\rlap/c}} %(Cents sign from the TeXBook.)

\begin{center}
\begin{tabular*}{0.75\textwidth}{@{\extracolsep{\fill}}rlrlrlrl}
 64 & \emph{space} &  129  &  (\verb|a|)  &  193  &  \verb|A|  &  240  &  \verb|0| \\
    &              &  130  &  (\verb|b|)  &  194  &  \verb|B|  &  241  &  \verb|1| \\
 74 &  (\cents)    &  131  &  (\verb|c|)  &  195  &  \verb|C|  &  242  &  \verb|2| \\
 75 &  \verb|.|    &  132  &  (\verb|d|)  &  196  &  \verb|D|  &  243  &  \verb|3| \\
 76 &  \verb|<|    &  133  &  (\verb|e|)  &  197  &  \verb|E|  &  244  &  \verb|4| \\
 77 &  \verb|(|    &  134  &  (\verb|f|)  &  198  &  \verb|F|  &  245  &  \verb|5| \\
 78 &  \verb|+|    &  135  &  (\verb|g|)  &  199  &  \verb|G|  &  246  &  \verb|6| \\
 79 &  \verb+|+    &  136  &  (\verb|h|)  &  200  &  \verb|H|  &  247  &  \verb|7| \\
 80 &  \verb|&|    &  137  &  (\verb|i|)  &  201  &  \verb|I|  &  248  &  \verb|8| \\
    &              &       &              &       &            &  249  &  \verb|9| \\
 90 &  (\verb|!|)  &  145  &  (\verb|j|)  &  209  &  \verb|J|  &       &           \\
 91 &  \verb|$|    &  146  &  (\verb|k|)  &  210  &  \verb|K|  &       &           \\
 92 &  \verb|*|    &  147  &  (\verb|l|)  &  211  &  \verb|L|  &       &           \\
 93 &  \verb|)|    &  148  &  (\verb|m|)  &  212  &  \verb|M|  &       &           \\
 94 &  \verb|;|    &  149  &  (\verb|n|)  &  213  &  \verb|N|  &       &           \\
 95 &  $\lnot$     &  150  &  (\verb|o|)  &  214  &  \verb|O|  &       &           \\
 96 &  \verb|-|    &  151  &  (\verb|p|)  &  215  &  \verb|P|  &       &           \\
 97 &  \verb|/|    &  152  &  (\verb|q|)  &  216  &  \verb|Q|  &       &           \\
    &              &  153  &  (\verb|r|)  &  217  &  \verb|R|  &       &           \\
107 &  \verb|,|    &       &              &       &            &       &           \\
108 &  \verb|%|    &  162  &  (\verb|s|)  &  226  &  \verb|S|  &       &           \\
109 &  \verb|_|    &  163  &  (\verb|t|)  &  227  &  \verb|T|  &       &           \\
110 &  \verb|>|    &  164  &  (\verb|u|)  &  228  &  \verb|U|  &       &           \\
111 &  \verb|?|    &  165  &  (\verb|v|)  &  229  &  \verb|V|  &       &           \\
    &              &  166  &  (\verb|w|)  &  230  &  \verb|W|  &       &           \\
122 &  \verb|:|    &  167  &  (\verb|x|)  &  231  &  \verb|X|  &       &           \\
123 &  \verb|#|    &  168  &  (\verb|y|)  &  232  &  \verb|Y|  &       &           \\
124 &  \verb|@|    &  169  &  (\verb|z|)  &  233  &  \verb|Z|  &       &           \\
125 &  \verb|'|    &       &              &       &            &       &           \\
126 &  \verb|=|    &       &              &       &            &       &           \\
127 &  \verb|"|    &       &              &       &            &       &           
\end{tabular*}
\end{center}  %$


%------------------------------------------------------------------------------------------------------

\newpage
\label{Syntactic Entities}
\printindex{syntax}{Syntactic Entities}

\label{Special}
\printindex{special}{Words with Special Meanings in Algol W}

\end{document}

